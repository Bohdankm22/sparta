\htmlhr
\chapter{Introduction\label{introduction}}

SPARTA is a research project at the University of Washington funded by the DARPA
 Automated Program Analysis for Cybersecurity (APAC) program.


SPARTA aims to detect certain types of malware in Android applications, or
to verify that the app contains no such malware.  SPARTA's verification
approach is type-checking:  the developer states a security property,
annotates the source code with type qualifiers that express that security
property, then runs a pluggable type-checker~\cite{PapiACPE2008,DietlDEMS2011} to verify the type
qualifiers (and thus to verify that the program satisfies the security
property).


%% This feels premature; readers don't know why or what parts of the
%% Checker Framework Manual to read.
% The SPARTA toolset is largely built upon 
% the Checker Framework. 
% For more information on pluggable type-systems, please consult the Checker Framework Manual at 
% \url{http://types.cs.washington.edu/checker-framework/}.  


You can find the latest version of this manual in the \<sparta-code> version
control repository, in directory \<sparta-code/docs>.  Alternately, you can
find it in a SPARTA release at
\url{http://types.cs.washington.edu/sparta/release/}, though that may not
be as up-to-date.

\todo{  we talk about the analyst before explaining what we mean by analyst.}

\section{Overview:  malware detection and prevention tools}

The SPARTA toolset contains two types of tools:  reverse engineering tools to find potentially
dangerous code in an Android app, and a tool to statically verify
information flow properties.

The reverse engineering tools to find potentially dangerous code can be run on
arbitrary unannotated Android source code.  Those tools give no guarantees,
however; they just direct the analyst's attention to suspicious locations
in the source code.

By contrast, the tools to statically verify information flow require a person to write
the information flow properties of the program, primarily as source code
annotations.  For instance, an object  that contains data that came from the
camera and is destined for the network would be annotated with 

\begin{Verbatim}
@Sources(CAMERA) @Sinks(NETWORK)
\end{Verbatim}



There are two different scenarios in which the SPARTA tools might be used.
\todo{This seems a little too verbose.}

\begin{itemize}
\item
  Ideally, the application vendor, who understands the source code,
  writes information flow annotations such as \<@Sources> in the source
  code, iterating until the static information flow tool issues no warnings.

  In this case, the analyst merely re-runs the static information flow tool
  to confirm the vendor's work.  This shows that there are no undesired
  information flows in the program.

  \chapref{analyze-annotated-app} explains how to use the SPARTA tools for
  this scenario.

\item
  If the application vendor delivers an unannotated program, then the
  analyst must understand the program well enough to annotate it and then
  annotate it.

  In this case, it is most efficient to first run the reverse engineering
  tools to detect suspicious code.  Those tools might reveal unacceptable
  code: either malware or code that the vendor
  should rewrite in a clearer or safer way.  If the suspicious code
  detection tools do not reveal problems so severe that the app should be
  rejected, then they help to guide the next step.   The analyst writes
  information flow annotations and runs the information
  flow tool until either the analyst has found a vulnerability or the lack
  of tool warnings indicates there is no vulnerability.

  \chapref{analyze-unannotated-app} explains how to use the SPARTA tools for
  this scenario.
\end{itemize}


\section{In case of trouble\label{sec:incaseoftrouble}}

\begin{sloppypar}
If you have trouble, please email either
\<sparta@cs.washington.edu>
(developers mailing list) or
\<sparta-users@cs.washington.edu> (users
mailing list) and we will try to help.
\end{sloppypar}




%%% Local Variables: 
%%% mode: latex
%%% TeX-master: "manual"
%%% TeX-command-default: "PDF"
%%% End: 

%  LocalWords:  Cybersecurity app SPARTA's sparta Sources
