\htmlhr
\chapter{Introduction\label{introduction}}

SPARTA is a project at the University of Washington funded by the DARPA
APAC (Automated Program Analysis for Cybersecurity) program.


SPARTA aims to detect certain types of malware in Android applications, or
to verify that the app contains no such malware.  SPARTA's verification
approach is type-checking:  the developer states a security property,
annotates the source code with type qualifiers that express that security
property, then runs a pluggable type-checker~\cite{PapiACPE2008,DietlDEMS2011} to verify that the type
qualifiers are right (and thus that the program satisfies the security
property).


The checker-framework is a pluggable type-checker that is used extensively in the SPARTA project. 
For more information on pluggable type-systems, please consult the checker-framework manual at 
\url{http://types.cs.washington.edu/checker-framework/}.  


Future updates to this manual will be posted to the public SPARTA project webpage at
\url{http://www.cs.washington.edu/sparta/}.


\section{In case of trouble}

If you have trouble, please email either\\
\<sparta@cs.washington.edu>
(developers mailing list) or\\
\<sparta-users@cs.washington.edu> (users
mailing list) and we will try to help.





