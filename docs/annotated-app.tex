\htmlhr
\chapter{How to Analyze an Annotated App\label{analyze-annotated-app}}

If you are presented with an annotated app, you can confirm the work of the
person who did the annotation by answering affirmatively three questions.

\begin{enumerate}
\item Does the Flow Checker produce any errors or warnings?
\item Does the flow-policy file match the application description?
\item Does the justification for each \<@SuppressWarnings> make sense?
\end{enumerate}

\section{Run the Flow Checker}

Run the Flow Checker (\chapref{flow-checker}) to ensure that there is no
data flow in the application beyond what is expressed in the given flow
policy:

\begin{alltt}
ant -DflowPolicy=myflowpolicy flowtest
\end{alltt}

If the Flow Checker produces any errors or warnings, then the app has not
been properly annotated and should be rejected.

\section{Review the Flow Policy}
Does the flow-policy file match the application description? There should
not be any flows that are not explained in the description.  These flows
may be explicitly stated, such as ``encrypt and sign messages, send them
via your preferred email app.''  Or a flow may only be implied, such as
``This Application allows the user to share pics with their contacts.''  In
the first example, you would expect an EMAIL sink to appear somewhere in
the policy file. In the second, ``share'' could mean a you would see a Flow
Sink of EMAIL, SMS, INTERNET, or something else.  Flows that are only
implied in the description could be grounds for rejection if the
description is too vague.

\section{Review \<@SuppressWarnings> Justifications}
Does the justification for every \<@SuppressWarnings> make sense? Search
for every instance of \<@SuppressWarnings("flow")> and read the
justification comment.  Compare the justification to the actual code and
determine if it make sense and should be allowed.  No justification comment
could be grounds for rejection.
  

%%% Local Variables: 
%%% mode: latex
%%% TeX-master: "manual"
%%% End: 
