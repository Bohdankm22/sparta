\htmlhr
\chapter{SPARTA internals\label{sparta-internals}}

This document contains details that are only relevant for
people inside the SPARTA team at UW.

The source code for the SPARTA checkers is in a Mercurial repository at
\</projects/swlab1/darpa-apac/sparta-code>.

To get a copy do:

\begin{alltt}
$ hg clone ssh://YOURID@SERVERNAME//projects/swlab1/darpa-apac/sparta-code
\end{alltt}
% make Emacs $ happy

To push your changes to the repository you need to be in sparta
group. Contact Werner or Mike to get the permission.

(Along with sparta-code, you may be interested in the sparta-meetings
and apac-meetings repositories for general information on the SPARTA
project.)

Note that SPARTA as well as the Checker Framework are evolving
rapidly.
Thus you may need to pull and update with updated source code and
rebuild the projects:

\begin{alltt}
$ hg fetch
\end{alltt}
% make Emacs $ happy

Test applications are stored in
\</projects/swlab1/darpa-apac/sparta-subjects>


Try to run ant in sparta-subjects/Sky 

\begin{alltt}
$ ant flowtest
\end{alltt}
% make Emacs $ happy

If it gives results like this, you're ready to work on annotating!

\begin{alltt}
[jsr308.javac] /home/syhan/jsr308/sparta-subjects/Sky/src/org/jsharkey/sky/WebserviceHelper.java:308: error: incompatible types.
[jsr308.javac]             HttpGet request = new HttpGet(String.format(WEBSERVICE_URL, lat, lon, days));
[jsr308.javac]                                                        ^
[jsr308.javac]   found   : @FlowSinks @FlowSources String
[jsr308.javac]   required: @sparta.checkers.quals.FlowSinks({sparta.checkers.quals.FlowSinks.FlowSink.NETWORK}) @FlowSources String
\end{alltt}

If you want to add a new application, put it under the
\<sparta-subjects> directory.

You may need to get Android source code to get sense of what api returns (or
gets) what type of data. See \url{http://source.android.com/source/index.html}
You can find the list of all apis from the android source code in 
\<frameworks/base/api/15.txt> - api list for api version 15 (Android 4.0.3)
Accessing resource is closely related to android permissions (some of the
resources are not protected with permissions though). Android permission list
is at:
\url{http://developer.android.com/reference/android/Manifest.permission.html}
Hints to add annotations could be permissionmap (which permission is required
to call which functions):
\url{http://www.android-permissions.org/permissionmap.html}


Adding Annotations
There are two parts in adding annotations. 
Firstly, we need to annotate the flow stub file (\<flow.astub>) in
\<sparta-code>.
It's used for annotating Android APIs. For instance, 

\begin{alltt}
package android.telephony;

class TelephonyManager {
    public @FlowSources(sparta.checkers.quals.FlowSources.FlowSource.PHONE_NUMBER) String getLine1Number();
    public @FlowSources(sparta.checkers.quals.FlowSources.FlowSource.IMEI) String getDeviceId();
}
\end{alltt}

The above annotates two methods in class \<TelephonyManager>.
It means that the \<getLine1Number> function returns a String which is
phone number.  For more examples, look into the \<flow.astub> file.

FlowSources specifies data sources such as phone number, location, and
etc. FlowSinks specifies sinks, such as files, network, and so on.
The types of FlowSources and FlowSinks are listed in the
\<FlowSources.java> and \<FlowSinks.java> files in \<sparta.checkers.quals>.


The second part is annotating the Android application to match
the annotations specified in the stub file. Run flowtest and see if
any incompatible types are shown and iteratively add more annotations
to the application.
The Android API specifications in flow.astub is currently
incomplete. As you experience new APIs, extend the specifications.
