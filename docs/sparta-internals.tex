\htmlhr
\chapter{SPARTA internals\label{sparta-internals}}

This document contains details that are only relevant to
SPARTA developers.  Also see the developer-docs directory in the sparta-code repository. 
Join the SPARTA dev list here:
\url{https://mailman.cs.washington.edu/mailman/listinfo/sparta} 

\section{Repositories\label{sec:sparta-repositories}}
The SPARTA team uses several Mercurial repositories.
Two on DADA:
\begin{itemize}
\item[] \<sparta-code> for the source code of the SPARTA toolset
\item[] \<sparta-sample-apps> for test applications (case studies)
\end{itemize}

\begin{alltt}
hg clone https://dada.cs.washington.edu/hgweb/<name>
\end{alltt}

To access  to the DADA repositories  you need to be in 
the \<sparta> web group. (See \url{https://wasp.cs.washington.edu/Internal/hg.html})

And several via SSH
\begin{itemize}
\item[] \<sparta-meetings> for notes about UW team meetings
\item[] \<apac-meetings> for notes about DARPA meetings
\item[] \<apac-engamentX> where X is 0-2, for APAC engagement applications
\end{itemize}


\begin{alltt}
hg clone ssh://buffalo.cs.washington.edu//projects/swlab1/darpa-apac/<name>
\end{alltt}


To access  to the filesystem repositories  you need to be in 
the \<sparta> Unix group. Contact Mike to get the permission.  


Also, if you are going to push changes, please add a \<.hgrc> file 
to your home directory on the server.  
The \<.hgrc> file should contain:

\begin{verbatim} 
[trusted]
users = wmdietl
groups = sparta
\end{verbatim}

This allows emails to be sent when you push changes. 

Note that SPARTA as well as the Checker Framework are evolving
rapidly.
Thus you should periodically get the latest version of the source code (by
running \<hg fetch>) and rebuild the projects.

\section{Cheat sheet for building dependancies\label{sec:cheat-sheet}}

The instructions in this section are only provide as a cheat sheet.  Please see the relevant manuals for
more information and complete instructions.

\subsection{Repositories}
Clone the following into a directory called jsr308:

\begin{alltt}
hg clone https://code.google.com/p/jsr308-langtools/ jsr308-langtools
hg clone https://code.google.com/p/checker-framework/ checker-framework
hg clone https://code.google.com/p/annotation-tools/ annotation-tools
hg clone https://plume-bib.googlecode.com/hg/ bib  
\end{alltt}

 No password for cloning/pulling, will need google account for pushing.
 
 \subsection{Building}
 
 \paragraph{JSR308 langtools}
 \begin{alltt}
cd $JSR308/jsr308-langtools/make
ant clean clean-and-build-all-tools
export PATH=$JSR308/jsr308-langtools/dist/bin:${PATH}
\end{alltt}

 \paragraph{Annotation Tools}
\begin{alltt}
cd $JSR308/annotation-tools
ant clean
ant
\end{alltt}

 \paragraph{Checker Framework}
\begin{alltt}
cd $JSR308/checker-framework/
ant clean
ant 
\end{alltt}

To build the Checker Framework without rebuilding the annotated JDKs:
\begin{alltt}
ant bindist-nojdk
\end{alltt}



\section{Required Software\label{sec:software}}

\subsection{Ant}
\url{http://ant.apache.org/bindownload.cgi}
\subsection{Android SDK}
\url{http://developer.android.com/sdk/index.html}
Install android-15 either in your IDE or from the command line run android
\subsection{\LaTeX}
We use \LaTeX plus Makefiles to create our webpages and manuals. 
\url{http://latex-project.org/ftp.html}
You also need \hevea \url{http://hevea.inria.fr/}
You may have to copy hevea.sty into sparta-code/docs 
New to  \LaTeX?   \url{http://en.wikibooks.org/wiki/LaTeX/Introduction} 

\subsection{IDE (Optional)}
Install your favorite.  Eclipse or VIM/EMACS are the most popular 

\subsection{Eclipse Plugins (Optional)}
Help -> Install New Software\\
Android ADT Plugin: Add new site: \url{https://dl-ssl.google.com/android/eclipse}\\
MercurialEclipse: Add new site: \url{http://cbes.javaforge.com/update} \\
Checker Framework Plugin: \url{http://types.cs.washington.edu/checker-framework/eclipse}\\


\subsection{Android Source code\label{sec:android-source-code}}


You may need to get Android source code to get sense of what API returns (or
gets) what type of data. See \url{http://source.android.com/source/index.html}
You can find the list of all APIs from the Android source code in 
\<frameworks/base/api/15.txt> - api list for api version 15 (Android 4.0.3)
Accessing resource is closely related to Android permissions (some of the
resources are not protected with permissions though).
The Android permission list is at:
\url{http://developer.android.com/reference/android/Manifest.permission.html}.
Hints to add annotations could be permissionmap (which permission is required
to call which functions):
\url{http://www.android-permissions.org/permissionmap.html}.

\section{Pushing Changes \label{pushing-changes}}

Typically for your first commit or for a large change, we review the changes first before you push.  After the review, there will likely be several changes you will need to make before you may push your changes.  

Before your code is reviewed, the follow should be true:

  \begin{itemize}
\item Your code is well formatted\footnote{ We following the Oracle code conventions: \url{http://www.oracle.com/technetwork/java/javase/documentation/codeconvtoc-136057.html}. You can also find an Eclipse formatter in the eclipse directory of sparta-code.}
\item You have comments on at least all public methods and classes.
\item Your code builds (make sure you fetch or pull/update checker framework, langtools and sparta-code first)
\item Your code pass ant all-tests
\item You added some tests JUnit tests.  (See sparta-code/tests/)
\item You updated the manual to reflect your changes. 
  \end{itemize}

Once you are ready for a review, send a patch or diff of the changes to be reviewed along with some words describing your changes.  To make a patch first \<hg add> all new files then follow the instructions at this link: 
\url{http://mercurial.selenic.com/wiki/QuickStart#Exporting_a_patch}

\section{Eclipse Instructions for SPARTA development\label{eclipse-instructions}}

Eclipse is the dominate IDE in our group.  Typically, Eclipse is used to navigate and debug the code but not build the tools.  This sections explains how to develop the 
SPARTA tools using Eclipse.  

\paragraph{Setup}


\begin{itemize}
\item Follow Section 25.3 from the Checker Framework manual 
(Building from source.)
\item As described in the installation instructions, set the \<CHECKERFRAMEWORK>
environment variable to \<.../checker-framework/>
\item Import the following projects: \<jsr308-langtools,
sparta-code, checkers, dataflow, framework, javacutil, stubparser>. The last 5 are  on the
\<checker-framework> folder.
\item The \<checkers> project should import \<javaparser> and \<jsr308-langtools>. 
The \<sparta-code> project should import \<checkers> and \<jsr308-langtools>.  
\item Select  \<Window $\rightarrow$ Preferences $\rightarrow$ Java 
$\rightarrow$ Installed JREs> and add the binaries \<.../checker/dist/javac.jar> 
and \<.../checker/dist/jdk7.jar> to the default JRE you are using.
\item Clean the projects on Eclipse. There should be no errors. 
\end{itemize}

\paragraph{Using tools on Eclipse}

\begin{itemize}
\item Import an Android Project to Eclipse. 
\item Open the build path of this project and add the sparta-code project.
\item Every .java file that will be annotated needs to contain the 
following imports:
  \begin{itemize}
    \item \<import  sparta.checkers.quals.Sink;>
    \item \<import  sparta.checkers.quals.Source;>
    \item \<import static sparta.checkers.quals.FlowPermission.*;>
  \end{itemize}
\item Select \<Run $\rightarrow$ External Tools $\rightarrow$ External Tools Configuration> 
and create a new \<Ant Build>.
\item Select the \<build.xml> file as the \<Buildfile> and you can use the SPARTA tools by passing
Arguments. For example: \<check-flow>.
\end{itemize}


\section{Helpful links\label{dev-links}}

\begin{itemize}
\item \url{http://types.cs.washington.edu/sparta/release/} (Do not publish)
\item \url{http://types.cs.washington.edu/checker-framework/  }

\item \url{http://homes.cs.washington.edu/~mernst/advice/version-control.html}
\item \url{https://wasp.cs.washington.edu/Internal/joining-plse.html}
\end{itemize}

Join the progress report mailing list, \url{https://mailman.cs.washington.edu/mailman/listinfo/progress-reports}.
 Instructions for writing progress reports can be found here, \url{http://homes.cs.washington.edu/~mernst/advice/progress-report.html}.

SPARTA on Jenkins (continuous integration server) \url{http://buffalo.cs.washington.edu:8080/view/SPARTA/}

Create a username for Jenkins \url{http://buffalo.cs.washington.edu:8080/securityRealm/addUser}

Checker Framework mailing list: \url{https://groups.google.com/forum/?fromgroups#!forum/checker-framework-dev}

Android permission maps:
\begin{itemize}
\item \url{http://www.android-permissions.org/}
\item \url{http://pscout.csl.toronto.edu}
\end{itemize}

%%% Local Variables: 
%%% mode: latex
%%% TeX-master: "manual"
%%% TeX-command-default: "PDF"
%%% End: 

%  LocalWords:  sparta apac hg jsr308 HttpGet WEBSERVICE lon Sink
%  LocalWords:  Source hgrc
