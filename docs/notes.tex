\htmlhr
\chapter{Notes\label{notes}}

Usage notes for the SPARTA information flow type system. This document
is meant to provide a reference for the specifics of the annotation
process, specifically regarding syntax in various situations and how
qualifiers are resolved.

[Note: \<FlowSource{}> is short for empty flow sources. \<@FlowSources...> is
just shorthand noting that the specifics aren't relevant.]

=====

JSR 308 extends the Java language to allow annotations in more
locations. Java 8 will include support for this extended syntax.
The Checker Framework builds on a version of OpenJDK that already
supports this syntax.
However, existing compilers do not support this syntax and will raise
an error. This is an issue in particular if you want to analyze
applications in Eclipse.

Eclipse accepts annotations only in the locations supported by Java
1.5, that is, only declaration locations; examples are fields, local
variables, parameters, and return types.

Eclipse won't accept annotations in locations that were added in
JSR 308; added locations include type arguments, object creation,
casts, type parameter bounds, and others.
To avoid syntax errors from Eclipse or other Java compilers, you need
to put such annotations in comments.

Single qualifiers can use the ``annotation-in-comment'' syntax
\</*@X*/>. Examples:

\begin{itemize}
\item Generics:
	\|List</*@FlowSources...*/ String>|

\item Object creation:
	\|new /*@FlowSinks...*/ Object();|

\item Casts:
	\|return (/*@FlowSources...*/ double) x;|
\end{itemize}

Note that only a single annotation can be in a comment; for
sequences, use multiple comments, as in \|/*@X*/ /*@Y*/|.

Also, in Java 8, method receivers can be explicitly given as a \<this>
parameter.
A separate comment syntax can be used to hide method receivers from
Eclipse, while still keeping them available to the Checker Framework:

\begin{alltt}
	public Class(/*>>> @FlowSinks(FlowSink.NETWORK) Class this, */ String paramTwo) {}
\end{alltt}

When using annotations in comments, Eclipse might warn about unused
imports for these annotations. To prevent such warnings, similarly put
such imports into comments:

\begin{alltt}
/*>>>
imports sparta.checkers.quals.*;
*/
\end{alltt}

Sometimes method type argument inference does not interact well with
type qualifiers. In such situations, you might need to provide
explicit method type arguments, for which the syntax is as follows:

\begin{alltt}
	Collections.</*@FlowSources({...})*/ Object>sort(l, c);
\end{alltt}

=====

To support adding annotations to libraries, we provide special .astub
files that contain the annotations on the signatures of such classes.
These files contain Java classes, but provide a few special features:
\begin{itemize}
\item they can contain multiple packages and multiple public classes in
 one file,
\item methods do not need to provide implementations,
\item the return type of a method does not need to match the real method,
 e.g. using java.lang.Object is valid for every method.
\end{itemize}

There are two temporary difference when annotating stub files, rather
than source code:
\begin{itemize}
\item the receiver is written after the method parameter list, instead of
as an explicit first parameter.
That is, instead of

\begin{alltt}
     returntype methodname(@Annotations C this, params);
\end{alltt}

in a stub file one has to write

\begin{alltt}
     returntype methodname(params) @Annotations;
\end{alltt}

\item enum constants in annotations need to be fully qualified. TODO: example
\end{itemize}


The constructor:
	mark as "@Annotations ClassName(params);" as if it was returning itself, rather than being a receiver.
TODO: this is not special to the stub file; this should be moved
	somewhere else.

======

TODO: provide type qualifier hierarchy, general overview, list of type
systems, ...

======

Default annotations at various places (fields vs local variables):
-Fields by default have no FlowSource or FlowSink. Local variables have FlowSource.ANY and FlowSink{}.


=====

\section{Miscellaneous explanations}

\begin{itemize}
\item It is safe to mark a \<@FlowSource{}> variable as
 \<@FlowSource{something}>, because we're just noting that it could have
 come from the source. It is NOT safe to do the same to a \<FlowSink{}>
 variable, because that would go from saying it CANNOT go to a sink to
 it MAY go to a sink. If you want to do so, mark it with
 \<@SuppressWarnings("flow")> before the annotation, and add a comment
 explaining why this annotation is safe.

\item For methods like \<equals()> and \<toString()> that are inherited from
 \<Object>, we require the Object class's annotations on those to be the
 most general possible, so that overriding methods can restrict the
 annotations further. Thus, they return FlowSource.ANY and
 FlowSink{}. Currently I have most toString() calls overwritten with a
 @SuppressWarnings("flow") and a comment explaining that. 

\item String literals [TODO: and all primitive literals? 1.0, true, 3, 'c',
 etc] have a default FlowSource{} FlowSink{}. A variable declaration:

 @FlowSink{something} String realString = "asdf";

 will raise an error, because an unqualified type flows to a sink.
 Add a @SuppressWarnings("flow"), or split it up into two
 lines, a @Suppressed '@FlowSink{something} String tempString =
 "asdf";' and then assign realString = tempString;

\item Adding on to that, if you try to compute a @FlowSource{something} int
 with a int literal (0 or 100, etc) it will resolve to another
 @FlowSource{something} int. If you try the same with a
 @FlowSink{something} int with an int literal, it will result in
 @FlowSink{} int because it's not safe to add a flowsink to the int
 literal. To bypass this, you must declare a local variable with
 "@SuppressWarnings("flow") @FlowSink{something} int tempZero = 0;"
 and then use the local variable in the calculation.
 Again document why the warning can be suppressed.

\item @SuppressWarnings("flow") cannot be used on assignments, only at
 variable or method declarations. This is relevant for local
 variables.
 The following is legal:

\begin{alltt}
	@SuppressWarnings("flow") // Explain why suppression is sound.
	@FlowSource... String a = method();
\end{alltt}

 The following is not legal, because the @SuppressWarnings is not on a
 declaration:

\begin{alltt}
	@FlowSource... String a; 
	@SuppressWarnings("flow") // Explain why suppression is sound.
	a = method();
\end{alltt}

 To get around this, add a new local variable with the proper
 annotations and suppress on that declaration, then assign the new local
 variable to the variable you originally wanted to assign to:

\begin{alltt}
	String a;
	@SuppressWarnings("flow") // Explain why suppression is sound.
	@FlowSource... String b = method();
	a = b;
\end{alltt}

\end{itemize}

-[TODO: Explain "top types" and the type hierarchy more.]
-[TODO: Explain flow inference.]
-[TODO: Explain polymorphic qualifiers.]
