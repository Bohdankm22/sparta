\htmlhr
\chapter{Notes\label{notes}}

This chapter is currently disorganized notes that will be incorporated into
either this manual or the Checker Framework manual.

[Note: \srcsome{} is shorthand noting that the specific flow
properties aren't relevant.]


\section{JSR 308 and Eclipse\label{sec:jsr308}}

JSR 308 extends the Java language to allow annotations in more
locations. Java 8 will include support for this extended syntax.
The Checker Framework builds on a version of OpenJDK that already
supports this syntax.
However, existing compilers do not support this syntax and will raise
an error. This is an issue in particular if you want to analyze
applications in Eclipse.

Eclipse accepts annotations only in the locations supported by Java
1.5, that is, only declaration locations; examples are fields, local
variables, parameters, and methods (return type annotations go in the same
place).

Eclipse won't accept annotations in locations that were added in
JSR 308; added locations include type arguments, object creation,
casts, type parameter bounds, and others.
To avoid syntax errors from Eclipse or other Java compilers, you need
to put such annotations in comments.
The SPARTA tools will interpret them, but Eclipse and other Java compilers
will ignore them.

For details, see section ``Annotations in comments'' in the Checker Framework manual.



Sometimes method type argument inference does not interact well with
type qualifiers. In such situations, you might need to provide
explicit method type arguments, for which the syntax is as follows:

\begin{alltt}
	Collections.</*@FlowSources({...})*/ Object>sort(l, c);
\end{alltt}


%%% Local Variables: 
%%% mode: latex
%%% TeX-master: "manual"
%%% TeX-command-default: "PDF"
%%% End: 
