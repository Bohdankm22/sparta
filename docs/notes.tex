\htmlhr
\chapter{Notes\label{notes}}

This chapter is currently disorganized notes that will be incorporated into
either this manual or the Checker Framework manual.

[Note: \srcsome{} is shorthand noting that the specific flow
properties aren't relevant.]


\section{JSR 308 and Eclipse}

JSR 308 extends the Java language to allow annotations in more
locations. Java 8 will include support for this extended syntax.
The Checker Framework builds on a version of OpenJDK that already
supports this syntax.
However, existing compilers do not support this syntax and will raise
an error. This is an issue in particular if you want to analyze
applications in Eclipse.

Eclipse accepts annotations only in the locations supported by Java
1.5, that is, only declaration locations; examples are fields, local
variables, parameters, and return types.

Eclipse won't accept annotations in locations that were added in
JSR 308; added locations include type arguments, object creation,
casts, type parameter bounds, and others.
To avoid syntax errors from Eclipse or other Java compilers, you need
to put such annotations in comments.

Single qualifiers can use the ``annotation-in-comment'' syntax
\</*@X*/>. Examples:

\begin{itemize}
\item Generics:
	\|List</*@FlowSources(...)*/ String>|

\item Object creation:
	\|new /*@FlowSinks(...)*/ Object();|

\item Casts:
	\|return (/*@FlowSources(...)*/ double) x;|
\end{itemize}

Note that only a single annotation can be in a comment; for
multiple annotations, use multiple comments, as in \|/*@X*/ /*@Y*/|.

Also, in Java 8, method receivers can be explicitly given as a \<this>
parameter.
A separate comment syntax can be used to hide method receivers from
Eclipse, while still keeping them available to the Checker Framework:

\begin{alltt}
        public Class(/*>>> @FlowSinks(FlowSink.NETWORK) Class this, */ String paramTwo) {}
\end{alltt}


When using annotations in comments, Eclipse might warn about unused
imports for these annotations. To prevent such warnings, similarly put
such imports into comments:

\begin{alltt}
/*>>>
import sparta.checkers.quals.*;
*/
\end{alltt}


Sometimes method type argument inference does not interact well with
type qualifiers. In such situations, you might need to provide
explicit method type arguments, for which the syntax is as follows:

\begin{alltt}
	Collections.</*@FlowSources({...})*/ Object>sort(l, c);
\end{alltt}


\section{Annotating libraries}

To support adding annotations to libraries, we provide special .astub
files that contain the annotations on the signatures of such classes.
These files contain Java classes, but provide a few special features:
\begin{itemize}
\item they can contain multiple packages and multiple public classes in
 one file,
\item methods do not need to provide implementations,
\item the return type of a method does not need to match the real method,
 e.g. using java.lang.Object is valid for every method.
\end{itemize}

There are two temporary difference when annotating stub files, rather
than source code:
\begin{itemize}
\item the receiver is written after the method parameter list, instead of
as an explicit first parameter.
That is, instead of

\begin{alltt}
     returntype methodname(@Annotations C this, params);
\end{alltt}

in a stub file one has to write

\begin{alltt}
     returntype methodname(params) @Annotations;
\end{alltt}

\item enum constants in annotations need to be fully qualified.
For example, one has to write the following

\begin{alltt}
@FlowSources(sparta.checkers.quals.FlowSources.FlowSource.ANY)
\end{alltt}

even when the classes are correctly imported.

\end{itemize}


% The constructor:
% 	mark as "@Annotations ClassName(params);" as if it was returning itself, rather than being a receiver.
% TODO: this is not special to the stub file; this should be moved
% 	somewhere else.


\section{Suppressing warnings}

Warnings can be suppressed using the standard \<@SuppressWarnings>
annotation.
To suppress warnings from the Flow Checker, use
\<@SuppressWarnings("flow")>.

The \<@SuppressWarnings> annotation cannot be used on statements or
expressions, only on declarations.
Warnings should be suppressed in the smallest possible scope and
should always contain an explanation for why it is safe to ignore the
warning.

The general pattern is to introduce a new local variable and suppress
the warning on that local variable.
The following is legal example for this:

\begin{alltt}
	@SuppressWarnings("flow") // Explain why suppression is sound.
	@FlowSources(...) String a = method();
\end{alltt}

The following is not legal, because the @SuppressWarnings is not on a
declaration:

\begin{alltt}
	@FlowSources(...) String a; 
	@SuppressWarnings("flow") // Explain why suppression is sound.
	a = method();
\end{alltt}

