\htmlhr
\chapter{Flow Checker\label{flow-checker}}

This chapter gives a brief overview of the Flow Checker, one of
the main components of SPARTA\@.  The Flow Checker tracks information flow
through your program --- it gives a guarantee that there is no information
flow beyond what is expressed in type annotations.

The Flow Checker does pluggable type-checking of an information flow type
system.  It is implemented using the Checker Framework.  To better
understand pluggable type-checking and the Checker Framework, please
consult the Checker Framework manual at
\url{http://types.cs.washington.edu/checker-framework/}.  

\section{Overview}

You write the annotation \<@sparta.checkers.quals.FlowSources> on a variable's
type to indicate what sensitive sources can affect the variable's value.
You write the annotation \<@FlowSinks> to indicate where (part of) the
value might be output.

As an example, suppose there is a declaration
\begin{Verbatim}
  @FlowSources(LOCATION) double latitude;
\end{Verbatim}
\noindent
This indicates that the value of \<latitude> might have been derived from
location information.  It does not guarantee that the value came from
authentic GPS data, but only sets a bound on where the information might
have come from; this assignment is legal: \<latitude = 47.6097>.

Similarly, the type
\begin{Verbatim}
  @FlowSinks(NETWORK) String someData;
\end{Verbatim}
\noindent
marks that the string \<someData> might be output to the network, or a part
of it or its length might be output to the network.  It is also possible
that the data has already been output.

The argument to \<@FlowSources> (and \<@FlowSinks>) is an enum constant, or a
set of them to indicate that a value might combine information from
multiple sources (or might flow to multiple locations).


\section{Subtyping}

As with standard Java types, the type annotation hierarchy indicates which
assignments, method calls, and overriding are legal.


\begin{figure}
\includeimage{flowsources}{5cm}
\caption{Partial qualifier hierarchy for \<@FlowSources>.}
\label{fig:flowsources-hierarchy}
\end{figure}

\figref{fig:flowsources-hierarchy} shows part of the \<@FlowSources>
qualifier hierarchy.  The top type is \<@FlowSources(ANY)>, which is
shorthand for listing every source.  It would be legal to annotate every
variable in a program with \<@FlowSources(ANY)>, because every variable is
derived from some subset of all flow sources.  But, such an annotation
would be imprecise.  \<@FlowSources({})> is the bottom type, and may only
be applied to variables whose value does not depend on any sensitive
source.


\begin{figure}
\includeimage{flowsinks}{5cm}
\caption{Partial qualifier hierarchy for \<@FlowSinks>.}
\label{fig:flowsinks-hierarchy}
\end{figure}

\figref{fig:flowsinks-hierarchy} shows part of the \<@FlowSinks>
qualifier hierarchy.  The top type is \<@FlowSinks(\{\})>, which indicates
that the value is only used locally by the application and never flows to
any sensitive sink.  The bottom type, \<@FlowSinks(FlowSinks.ANY)>, is a
value that might be output to any location whatsoever; it can be thought of
as a completely public value.

Note the different subtyping behavior for sources and sinks.


\section{Defaults}

The default qualifiers are \<@FlowSinks({})> and \<@FlowSources({})>.  This
default applies to field and method signature types, manifest constants
(string literals, \<1.0>, \<true>, \<3>, \<'c'>,
\ldots), and all other locations except local variables.

This means that this variable declaration raises an error:

\begin{alltt}
@FlowSinks(NETWORK) String realString = "asdf";
\end{alltt}

The reason is that the right-hand side of the assignment has type
\<@FlowSinks({})>, which is a value that may not be output to any sink.
This is not a subtype of the left-hand type \<@FlowSinks(NETWORK)>, which
is allowed to flow to the network (but to no other sink).  If the given
string is permitted to flow to the network, then you can suppress the
warning by annotating the declaration with \<@SuppressWarnings("flow")>.


For \textbf{local variables}, the default is the top type for each
hierarchy, that is, any flow source and no flow sinks.  The Checker
Framework supports flow-sensitive type refinement.  Assignments (such as
initializers) are used to refine the type to a more precise one.  Thus, in
general you do not have to write type annotations on local variables.  For
details, see section ``Automatic type refinement (flow-sensitive type
qualifier inference)'' in the Checker Framework manual.


TODO: Are defaults for libraries different?


\section{Qualifier polymorphism}

To make the type system more expressive, the flow type systems support
qualifier polymorphism, via the type qualifiers \<@PolyFlowSources> and
\<@PolyFlowSinks>.
See section ``Qualifier polymorphism'' in the Checker Framework manual.

% These range over their respective type hierarchy and can be used to
% express type dependencies, for example, that the receiver and
% parameter types have to correspond.
% The invocation of a method is used to resolve polymorphic qualifiers
% to something concrete.


\section{API specifications}

File \<sparta-code/src/sparta/checkers/flow.astub> provides library
annotations.  See chapter ``Annotating libraries'' in the Checker Framework
manual.

Besides the qualifiers discussed so far, two additional declaration
annotations can be used here: \<@NoFlow> and \<@ConservativeFlow>.
These annotations can be used on any declaration: a method, a class,
or even a whole package.
For example, in the beginning of \<flow.astub> we declare that the
whole \<android> package should use conservative defaults.
More specific annotations given in the rest of the file override these
defaults.

Annotation \<@NoFlow> expresses that each method's return type and all
parameter types are \<@FlowSources({}) @FlowSinks({})>.  As with all
library annotations, it is trusted rather than checked.  Thus, it should be
used only if an external analysis has determined that it is correct for the
annotated method, class, or package.

Annotation \<@ConservativeFlow> expresses that each method should have the
most conservative possible annotations:
  \<@FlowSources({}) @FlowSinks(ANY)> on arguments, and
  \<@FlowSources(ANY) @FlowSinks({})> on return values.
[TODO: what for fields?]
This is so conservative that it is sure to cause a type-checking failure
whenever the method is used.
When the analyst encounters such type-checking errors, the analyst can
annotate the methods more appropriately.  This is a way of knowing when a
program uses a previously-unannotated library.


\section{Stricter tests}

The default checks ensure soundness of the flow qualifiers.
It ignores two possible sources of unsoundness: covariant array
subtyping and downcasts that are not checked at runtime.
After getting the basic checks passing, the stricter checks should be
enabled.
This two-phase approach was chosen to not interfere with
the annotation effort too severely and to give two separate phases of
the annotation effort.


When strict checks are turned on,
a cast \<(Object []) x>, were \<x> is of type \<Object>, will result
in a compiler warning:

\begin{alltt}
[jsr308.javac] ... warning: "@FlowSinks @FlowSources({FlowSource.ANY}) Object"
       may not be casted to the type "@FlowSinks @FlowSources Object"
\end{alltt}

The reason is that statically the component type of the array is
simply defaulted. There is no static knowledge about the actual
runtime values in the array and important flow could be hidden.
The analyst should argue why the downcast is safe.

Note that the main qualifier of a cast is automatically flow-refined
by the cast expression.


\medskip

Stricter checking also enforces invariant array subtyping, which is
needed for sound array behavior in the absence of runtime checks.
Flow inference automatically refines the type of array creation
expressions depending on the left-hand side.



\section{Miscellaneous}

Methods like \<equals()> and \<toString()> that are inherited from
\<Object> are the most general possible, so that overriding methods
can restrict the annotations further. Thus, they
return \<FlowSource.ANY> and no flow sinks.

Binary operations like string concatenation or integer addition,
result in the least upper bound of the two component types.


%% Adding on to that, if you try to compute a @FlowSource{something} int
%%  with a int literal (0 or 100, etc) it will resolve to another
%%  @FlowSource{something} int. If you try the same with a
%%  @FlowSink{something} int with an int literal, it will result in
%%  @FlowSink{} int because it's not safe to add a flowsink to the int
%%  literal. To bypass this, you must declare a local variable with
%%  "@SuppressWarnings("flow") @FlowSink{something} int tempZero = 0;"
%%  and then use the local variable in the calculation.
%%  Again document why the warning can be suppressed.


%%% Local Variables: 
%%% mode: latex
%%% TeX-master: "manual"
%%% TeX-command-default: "PDF"
%%% End: 

%  LocalWords:  FlowSinks someData FlowSources enum 5cm flowsources 'c'
%  LocalWords:  flowsinks realString asdf PolyFlowSources PolyFlowSinks
%  LocalWords:  sparta NoFlow ConservativeFlow astub TODO jsr308
%  LocalWords:  FlowSource
