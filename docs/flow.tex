\htmlhr
\chapter{Flow Checker\label{flow-checker}}

This chapter gives a brief overview of the Flow Checker, a type-checker that 
tracks information flow through your program.  It gives a guarantee that there
 is no information flow beyond what is expressed in the flow policy and type annotations.

The Flow Checker does pluggable type-checking of an information flow type
system.  It is implemented using the Checker Framework.  To better
understand pluggable type-checking and the Checker Framework, please
consult the Checker Framework manual at
\url{http://types.cs.washington.edu/checker-framework/}.  

\section{Overview}

You write the annotation \<@FlowSources> on a variable's
type to indicate what sensitive sources can affect the variable's value.
You write the annotation \<@FlowSinks> to indicate where (part of) the
value might be output.

As an example, suppose there is a declaration
\begin{Verbatim}
  @FlowSources(FlowSource.LOCATION) @FlowSinks(FlowSink.NETWORK) double latitude;
\end{Verbatim}
\noindent
The \<@FlowSources(FlowSource.LOCATION)> annotation indicates that the value of \<latitude> might have been derived from
location information.  It does not guarantee that the value came from
authentic GPS data, but only sets a bound on where the information might
have come from; this assignment is legal: \<latitude = 47.6097>.
Similarly, the annotation \<@FlowSinks(FlowSInk.NETWORK)> marks that the string \<latitude> 
might be output to the network.  It is also possible that the data has already been output.

The argument to \<@FlowSources> (and \<@FlowSinks>) is an enum constant, or a
set of them to indicate that a value might combine information from
multiple sources (or might flow to multiple locations).
FlowSources specifies data sources such as phone number, location, 
etc.  FlowSinks specifies sinks, such as files, network, and so on.
The types of FlowSources and FlowSinks are listed in the
\<FlowSources.java> and \<FlowSinks.java> files in \<sparta.checkers.quals>.






\section{Default Types\label{flow-defaults}}

\begin{table}[h]
    \begin{tabular}{| l | l |}
     \hline
    Fields Types &  \<@FlowSources @FlowSinks> \\ \hline
    Method Parameters &  \<@FlowSources @FlowSinks> \\ \hline
    Return Types &  \<@FlowSources @FlowSinks> \\ \hline
    \<null> &  \<@FlowSources @FlowSinks(FlowSink.ANY) >\\ \hline
    Literals & \<@FlowSources(FlowSource.LITERAL)@FlowSinks(FlowSink.CONDITIONAL)>\\ \hline
   Local Variables  &   \<@FlowSources(FlowSource.ANY) @FlowSinks> \\ \hline
   \<@FlowSources($\alpha$)>&\<@FlowSources($\alpha$)>\<@FlowSinks($\beta$)>,  $\beta$ is all flow sinks allowed to flow from a source in $\alpha$ \\ \hline
      \<@FlowSinks($\beta$)>&\<@FlowSources($\alpha$)>\<@FlowSinks($\beta$)>, $\alpha$ is all flow sources allowed to flow to a sink in $\beta$ \\ \hline
    \end{tabular}
    \caption{Default Types}\label{table:defaults}
\end{table}


To reduce the number of annotations needed, default types are used. Table \ref{table:defaults} 
shows some  the default types used by the flow checker. 


 The defaults are not applied if the programmer uses annotations.  For example, the parameter 
below is of type \<@FlowSources(FlowSource.LITERAL) @FlowSinks(FlowSink.Network)> rather
than  \<@FlowSources @FlowSinks>. Note that \<@FlowSources> means \<@FlowSources(\{\})>
 and \<@FlowSinks>  means \<@FlowSinks(\{\})>.
\begin{alltt}
public void sendToInternet(
     @FlowSources(FlowSource.LITERAL)@FlowSinks(FlowSink.Network) String message)\{...\}
\end{alltt} 

The value \emph{null} has the bottom type (@FlowSinks(FlowSink.ANY) @FlowSources), 
so that it can be assigned to any type. For local variables, the default applies to
the top level of the local variable type, but not to generic type arguments
and array elements, if any.

The Checker Framework supports flow-sensitive type refinement.  Assignments (such as
initializers) are used to refine the type to a more precise one.  Thus, in
general you do not have to write type annotations on local variables.  For
details, see section ``Automatic type refinement (flow-sensitive type
qualifier inference)'' in the Checker Framework manual.

In the case that the programmer only specifies flow sources, the flow sink is defaulted to be all 
sinks that the specified flow sources can flow to.
 In other words, if a type is annotated with 
\<@FlowSources($\alpha$)>, where $\alpha$ is a set of sources, then the flow sinks are the set 
$\beta$ where $\beta$ is the union of sinks B where A-\textgreater B and A is a flow source in 
$\alpha$.  For example, if the flow policy contains the following:

\begin{alltt}
  A->X,Y
  B->Y
  C->Y
\end{alltt}
  
\noindent 
then these are equivalent:

\begin{alltt}
  @FlowSources(A)                 ==   @FlowSources(A) @FlowSinks(\{X, Y\})
  @FlowSources(B)                 ==   @FlowSources(B) @FlowSinks(Y)
  @FlowSources(\{B,C\})             ==   @FlowSources(B,C) @FlowSinks(Y)
  @FlowSources(A) @FlowSinks(Y)   ==   @FlowSources(A) @FlowSinks(Y) 
\end{alltt}

Also, \<@FlowSources(A,B)>  is equivalent to 
\<@FlowSources(\{A,B\}) @FlowSinks(\{X,Y\})>, but it does not type check because B is not allowed
to flow to Y.  This means that \<@FlowSources(A,B)>  will not type check either.


Similarly, if the programmer only specifies flow sinks, the flow sources are defaulted to be all sources 
that are allowed to flow to  the specified sinks.  In other words, if a type is annotated with 
\<@FlowSinks($\beta$)>, where $\beta$ is a set of sinks, then the flow sources are the set 
$\alpha$ where $\alpha$ is the union of sources A where A-\textgreater B and B is a flow sink in 
$\beta$. For example, use the same policy file as above, the following are equivalent:

\begin{alltt}
  @FlowSinks(X)                   ==   @FlowSources(A) @FlowSinks(X)
  @FlowSinks(Y)                   ==   @FlowSources(\{A,B,C\}) @FlowSinks(Y)
\end{alltt}

\section{Flow Policy}
\label{sec:flowpolicy}

A flow policy is a list of the only flows that are permitted to occur in an application.
Flow policies are specified in a flow-policy file that is  a list 
 of flow source and flow sink pairs.  If a flow is not listed in the flow policy, then 
 it is forbidden to occur.  If no flow policy is specified, then no flows are permitted. 
 
\subsection{Semantic of a Flow Policy}
\label{sec:undsiredflows}
The Flow Checker guarantees that there is no information
flow except for what is explicitly permitted by the policy file. If a user writes a type that is
not permitted by the policy file, then the flow checker issues a warning
even if all types in program otherwise typecheck.

For example, this variable declaration

\begin{alltt}
@FlowSource(FlowSource.CAMERA) @FlowSink(NETWORK) Video video = ...
\end{alltt}

\noindent
is illegal unless the the policy file contains:

\begin{alltt}
CAMERA -> NETWORK
\end{alltt}

Here is another example.
The flow policy file contains:
\begin{alltt}
  ACCOUNTS      -> EXTERNAL_STORAGE, FILESYSTEM
  ACCELEROMETER -> EXTERNAL_STORAGE, FILESYSTEM, NETWORK
\end{alltt}

The following variable declarations are permitted:
\begin{alltt}
  @FlowSources(FlowSource.ACCOUNTS) @FlowSinks(FlowSink.EXTERNAL_STORAGE) Account acc = ...
  @FlowSources({FlowSource.ACCELEROMETER, FlowSource.ACCOUNTS})
  @FlowSinks({FlowSink.EXTERNAL_STORAGE, FlowSink.FILE_SYSTEM}) int accel = ...
\end{alltt}

The following definitions would generate ``forbidden flow'' errors:
\begin{Verbatim}
  @FlowSources(FlowSource.ACCOUNTS) @FlowSinks(@FlowSink.NETWORK) Account acc = ...
  @FlowSources({FlowSource.ACCELEROMETER, FlowSource.ACCOUNTS})
  @FlowSinks({FlowSink.EXTERNAL_STORAGE, FlowSink.FILESYSTEM, FlowSink.NETWORK})
\end{Verbatim}

\subsection{Syntax of a Flow Policy File}

Each line of a policy file specifies a source-to-sink(s) flow(s) that is permitted. So if 
A $\rightarrow$ B,C is listed in the file, then information from source A is allowed to flow
to sink B or sink C.  

For example,
\<MICROPHONE$\rightarrow$NETWORK> implies that
MICROPHONE data is always allowed to flow to NETWORK.  
The source must be a member of the enum
\<sparta.checkers.quals.FlowSources.FlowSource> and the sink must be a
member of the enum 
\<sparta.checkers.quals.FlowSinks.FlowSink>.  The source 
and sink names should not be preceded by the name of the enumeration which contains them.
ANY is allowed just as it is in @FlowSources and @FlowSinks, but empty, \{\}, is not allowed.

Multiple sources or sinks can appear on the same line if they are separated by commas. 
For example,
these two policy files have the same meaning:

\begin{alltt}
   MICROPHONE -> NETWORK, LOGCAT, DISPLAY
\end{alltt}
is equivalent to this policy file:
\begin{alltt}
   MICROPHONE -> NETWORK
   MICROPHONE -> LOGCAT
   MICROPHONE -> DISPLAY, NETWORK
\end{alltt}

The policy file may contain blank lines and comments that begin with 
a number sign (\<\#>) character.



\subsection{Using a flow-policy file}
To use a flow-policy file when invoking the Flow Checker from the
command line pass it the option:
\begin{alltt}
-AflowPolicy=\emph{mypolicyfile}
\end{alltt}

Or if you are using one of the ant targets, you can pass the option to ant:
\begin{alltt}
ant -DflowPolicy=\emph{mypolicyfile} flowtest
\end{alltt}

Remember, not specify a flow policy file is equivalent  to not allowing any flows. 

\section{Subtyping}

\begin{figure}[h]
\includeimage{flowsources}{5cm}
\caption{Partial qualifier hierarchy for \<@FlowSources>.}
\label{fig:flowsources-hierarchy}
\end{figure}

\begin{figure}[h]
\includeimage{flowsinks}{5cm}
\caption{Partial qualifier hierarchy for \<@FlowSinks>.}
\label{fig:flowsinks-hierarchy}
\end{figure}

As with standard Java types, the type annotation hierarchy indicates which
assignments, method calls, and overriding are legal. \figref{fig:flowsources-hierarchy} 
shows part of the \<@FlowSources>
qualifier hierarchy.  The top type is \<@FlowSources(ANY)>, which is
shorthand for listing every source.  It would be legal to annotate every
variable in a program with \<@FlowSources(ANY)>, because every variable is
derived from some subset of all flow sources.  But, such an annotation
would be imprecise.  \<@FlowSources(\ttcbs)> or \<@FlowSources()> is the
bottom type, and may only
be applied to variables whose value does not depend on any sensitive
source.


\figref{fig:flowsinks-hierarchy} shows part of the \<@FlowSinks>
qualifier hierarchy.  The top type is \<@FlowSinks(\ttcbs)> or
\<@FlowSinks()>, which indicates
that the value is only used locally by the application and never flows to
any sensitive sink.  The bottom type, \<@FlowSinks(FlowSinks.ANY)>, is a
value that might be output to any location whatsoever; it can be thought of
as a completely public value.

Note the different subtyping behavior for sources and sinks.




\section{Conditionals}

The Flow Checker considers conditional expressions to be a flow sink.  If a variable 
will be used in a conditional, then it must have a flow sink of CONDITIONAL.  By default, 
any source is allowed flow through a conditional.  That is to say that 
ANY -\textgreater CONDITIONAL is added to the flow policy by default. 

Conditionals are treated this way because they can leak information. For example, the given a flow policy of  USER\_INPUT -\textgreater FILESYSTEM, the following type checks.

\begin{alltt}
@FlowSources(FlowSource.USER_INPUT)@FlowSinks(FlowSink.FILESYSTEM)
int creditCard = getCCNumber();
final int MAX\_CC\_NUM=9999999999999999;
for (int i = 0 ; i < MAX\_CC\_NUM ; i++)\{
   if ( i == creditCard )
        sendToInternet(i);
\}
\end{alltt}

To catch this sort of information leak, pass
 \<-Alint=strict-conditional> to change the default from ANY -\textgreater CONDITIONAL to 
 LITERAL-\textgreater CONDITIONAL. 



\section{Empty Flow Sinks or Flow Sources}

Although @FlowSources(\{\}) and @FlowSinks(\{\}) are used in the default types, 
 the programmer should not annotate any types with these flows.  Every value 
 should either flow flow from a literal or from some sensitive source.  Likewise, every value
 must flow to a sensitive sink or to a conditional expression.  Any variable that does not
 have a flow source or a flow sink  does not actually affect the
  output of the program and should therefore be removed.  
  
  This may seem overly strict, but a variable without a flow source or flow sink that does affect
  the output of the program comes from an abuse of the type system.   Most likely  a variable with
   no source or sink would come from  an improperly suppressed warning. Therefore it is necessary to 
  not allow  flows from and to nowhere.  (The flow policy ensures this; see section 
  Section \ref{sec:flowpolicy})
  
  Note that this does not mean you must specify both a flow source annotation and 
  a flow sink annotation as explained in Section \ref{flow-defaults}.

 
 \section{Warning Suppression}
 
 Sometimes it might be necessary to suppress warnings or errors produced by the
  Flow Checker.  This can be done by using the  \<@SuppressWarnings("flow")> annotation
   on a declaration.  Because this annotation can be used to subvert the Flow Checker, its use
   is considered suspicious.  Anytime a warning or error is suppressed, you must 
   write a brief comment justifying the suppression.     \<@SuppressWarnings("flow")> should
   only be used if there is no way to annotate the code so that an error or warning does not occur. 
   Most programs should not suppress warnings. 
\section{API specifications}

File \<sparta-code/src/sparta/checkers/flow.astub> provides library
annotations.  You may need to enhance it, if you find that your application
uses APIs that are not yet annotated.  For details, see
section~\ref{flow-task-annotate-apis} of this manual, and also 
chapter
``Annotating libraries'' in the Checker Framework manual.

\section{Additional Annotations}
Besides the qualifiers discussed so far, four additional declaration
annotations can be used to annotate APIs:\<@PolyFlowSources>,
 \<@PolyFlowSinks>, \<@NoFlow>, and \<@ConservativeFlow>. 
 They change the default annotations (Section~\ref{flow-defaults}).
 
 These annotations can be used on any declaration: a method, a class,
or even a whole package.
For example, the beginning of \<flow.astub> declares that the
whole \<android> package should use conservative defaults.
More specific annotations given in the rest of the file override these
defaults.
 
\subsection{@PolyFlowSources  and @PolyFlowSinks}

To make the type system more expressive, the flow type systems support
qualifier polymorphism, via the type qualifiers \<@PolyFlowSources> and
\<@PolyFlowSinks>.  These are mostly used when annotating API's when the
 specific flow sources or flow sinks are not known or can vary. 
See section ``Qualifier polymorphism'' in the Checker Framework manual.  

% These range over their respective type hierarchy and can be used to
% express type dependencies, for example, that the receiver and
% parameter types have to correspond.
% The invocation of a method is used to resolve polymorphic qualifiers
% to something concrete.
\subsection{@NoFlow}
The a method or class annotated with \<@NoFlow> expresses that the 
method's return type and all parameter types are \<@FlowSources({}) @FlowSinks({})>.  As with all
library annotations, it is trusted rather than checked.  Thus, it should be
used only if an external analysis has determined that it is correct for the
annotated method, class, or package.

\subsection{@ConservativeFlow}

Annotation \<@ConservativeFlow> expresses that each method should have the
most conservative possible annotations:
  \<@FlowSources({}) @FlowSinks(ANY)> on arguments, and
  \<@FlowSources(ANY) @FlowSinks({})> on return values.
%[TODO: what for fields?]
This is so conservative that it is sure to cause a type-checking failure
whenever the method is used.
When the analyst encounters such type-checking errors, the analyst can
annotate the methods more appropriately.  This is a way of knowing when a
program uses a previously-unannotated library.




%TODO generate some more examples.
%\subsection{Examples}
%
%Below are some examples of policy files and code examples that
%illustrate their effects.  Only assignment operations are used
%but remember that the flow-policy applies to all information flows.
%Comments in the code examples describe a direct effect of the
%given policy file.  Fake FlowSource and FlowSink names are used
%to aid in clarity.
%
%Any variable definition marked "OK" indicates that this
%definition would  be allowed given the policy file
%(and consequently also allowed in the absence of any flow-policy
%file).  Any variable definition marked "FORBIDDEN" indicates that
%the definition is forbidden by the flow-policy file.
%
%Any assignment marked "OK" indicates that this assignment is valid
%(with or without the policy file).  An assignment
%marked "FORBIDDEN" indicates that the assignment will still fail
%with an error.




\section{Stricter tests}

By default, the flow checker is unsound.  After getting the basic checks to pass, the
 stricter checks should be enabled, by running \<ant -Dsound=true flowtest>.
This two-phase approach was chosen to reduce
the annotation effortand to give two separate phases of
the annotation effort.
 The sound checking enforces invariant 
array subtyping and type safety in downcasts.


When strict checks are turned on,
a cast \<(Object []) x>, were \<x> is of type \<Object>, will result
in a compiler warning:

\begin{alltt}
[jsr308.javac] ... warning: "@FlowSinks @FlowSources({FlowSource.ANY}) Object"
       may not be casted to the type "@FlowSinks @FlowSources Object"
\end{alltt}

The reason is that there is not way for the type-checker to verify
 the component type of the array. There is no static knowledge about the actual
runtime values in the array and important flow could be hidden.
The analyst should argue why the downcast is safe.

Note that the main qualifier of a cast is automatically flow-refined
by the cast expression.


\medskip

Stricter checking also enforces invariant array subtyping, which is
needed for sound array behavior in the absence of runtime checks.
Flow inference automatically refines the type of array creation
expressions depending on the left-hand side.

\section{Miscellaneous}

Methods like \<equals()> and \<toString()> that are inherited from
\<Object> are the most general possible, so that overriding methods
can restrict the annotations further. Thus, they
return \<FlowSource.ANY> and no flow sinks.

Binary operations like string concatenation or integer addition,
result in the least upper bound of the two component types.


%% Adding on to that, if you try to compute a @FlowSource{something} int
%%  with a int literal (0 or 100, etc) it will resolve to another
%%  @FlowSource{something} int. If you try the same with a
%%  @FlowSink{something} int with an int literal, it will result in
%%  @FlowSink{} int because it's not safe to add a flowsink to the int
%%  literal. To bypass this, you must declare a local variable with
%%  "@SuppressWarnings("flow") @FlowSink{something} int tempZero = 0;"
%%  and then use the local variable in the calculation.
%%  Again document why the warning can be suppressed.


%%% Local Variables: 
%%% mode: latex
%%% TeX-master: "manual"
%%% TeX-command-default: "PDF"
%%% End: 

%  LocalWords:  FlowSinks someData FlowSources enum 5cm flowsources 'c'
%  LocalWords:  flowsinks realString asdf PolyFlowSources PolyFlowSinks
%  LocalWords:  sparta NoFlow ConservativeFlow astub TODO jsr308
%  LocalWords:  FlowSource
