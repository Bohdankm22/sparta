\htmlhr
\chapter{Flow Checker\label{flow-checker}}

Each type consists of a flow source and a flow sink qualifier.

Each qualifier is a set of enum constants from the respective enum,
defined in \<sparta.checkers.quals.FlowSources.FlowSource> and
\<sparta.checkers.quals.FlowSinks.FlowSink>.


See \figref{fig:flowsources-hierarchy} for an example hierarchy of
source qualifiers and \figref{fig:flowsinks-hierarchy} for sink
qualifiers.


\begin{figure}
\includeimage{flowsources}{5cm}
\caption{Qualifier hierarchy for four possible flow sources.}
\label{fig:flowsources-hierarchy}
\end{figure}

\begin{figure}
\includeimage{flowsinks}{5cm}
\caption{Qualifier hierarchy for four possible flow sinks.}
\label{fig:flowsinks-hierarchy}
\end{figure}


Explain qualifier subtyping for each individual hierarchy using the examples.

Explain subtyping.

Explain @PolyFlowSources and @PolyFlowSinks.

Explain flow.astub.
Explain defaulting, @NoFlow, @ConservativeFlow.

A cast \<(Object []) x>, were \<x> is of type \<Object>, will result
in a compiler warning:

\begin{alltt}
[jsr308.javac] ... warning: "@FlowSinks @FlowSources({FlowSource.ANY}) Object"
       may not be casted to the type "@FlowSinks @FlowSources Object"
\end{alltt}

The reason is that statically the component type of the array is
simply defaulted. There is no static knowledge about the actual
runtime values in the array and important flow could be hidden.
Use \<-Dstricter> to turn on these stricter checks.
