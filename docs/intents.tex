\htmlhr

\newcommand{\tp}[1]{\mathit{typeof}}
\newcommand{\Fix}[1]{\textbf{[[}{\color{red} #1}\textbf{]]}}


\chapter{Handling Intents\label{intent-checker}}

Android intents are the standard inter-component communication
mechanism in Android.  They are used for communication within an app,
among apps, and with the Android system.  An Android component can
send or broadcast intents and can register interest in receiving
intents.

Intents can be seen as messages exchanged through components; the
payloads of these messages are maps.  Sensitive data can flow in and
out of intent objects.  Consequently, to prevent manifestations of
forbidden flows, \theFlowChecker{} needs to be aware of the
information flow types associated to the various data elements
encapsulated in intent maps.

\TheFlowChecker{} provides support to
ensure that the type of data stored within an intent conforms to the
declared type of the intent, even in the presence of implicit
invocation via intents.  It verifies that for any sending intent
method call and any receiving intent method declaration that can be
invoked by the call site, the intent argument of the caller is
compatible with that of the corresponding callee.  In addition, the
checker verifies that the intent payload is used consistently at each
site.

For simplicity in this manual, we abstract all methods that send intents as the
method sendIntent, and all methods that receive an intent as the method call
onReceive.

\section{Component map}
To precisely analyze the types of data sent through intents, our analysis
requires sendIntent calls to be matched to the declarations of the onReceive
methods they implicitly invoke. A sendIntent call may be paired
with more than one onReceive declaration. One such pair indicates that 
components a and b, possibly from different apps, may communicate through 
intents. 

A component map is used to represent the communication between components of an
application. A component map file expresses a 
component map, as a list of $\flow{\|sendIntent|}{\|onReceive|^*}$ pairs.

\subsection{Semantics of a Component map}
\TheFlowChecker{} assumes that there is no inter-component communication
except for what is explicitly described in the policy file. 

\subsection{Syntax of a Component map file}
Each line of a component map specifies the inter-component communication of
an application. For example, \code{ActivityA(Action1) -> ActivityB} implies that
every implicit intent sent from ActivityA with action \code{Action1} will
be sent to ActivityB. 

\subsection{Component map generation}
To generate a component map for an application, the generate-cm script, which 
can be found in the sparta-code source folder, is used.
It receives as argument the full path of the application's apk and the output
path of the component map. The second one is optional and the component map
will be generated in the same folder as the apk in case not informed.
\begin{alltt}
./generate-cm [full path to app's .apk] [output path]
\end{alltt}

\subsection{Using a Component map}
To use a component map file named \emph{component-map} 
when invoking \TheFlowChecker{} from the command line, move the file to the
source folder of the application and 
add the following line in the build.xml file of the application:
\begin{alltt}
<property name="componentMap" value="component-map"/>
\end{alltt}

\section{IntentMap and Extra annotations}
An Intent contains a map from string keys to arbitrary values. Data can be added
to this map by the sender with \<intent.putExtra("key",value)> and then
retrieved by the receiver with \<intent.getExtra("key",default)>. 
The \<@IntentMap> represents the types
of the key/value mappings that are permitted to be accessed. The Intent object
may contain more mappings than stated in its type, but those other keys may not
be accessed. The Intent object is not required to contain all the given
mappings, but if a key is present, then the type of the corresponding value must
be consistent with the stated type. 

An \<@IntentMap> annotation contains a set of \<@Extra> annotations. An
\<@Extra> annotation contains a key K and a type T. This means that the key K 
maps to a value of type T. In the case of the information flow system, the type T
represents data and constrains its possible flows. 

\subsection{putExtra and getExtra rules}

\subsubsection{putExtra}
Calls intent.putExtra(key,value) type-check if:
\begin{itemize}
\item
 key is in an \<@Extra> in the intent's \<@IntentMap> annotation.
\item
 value is a subtype of the type of the \<@Extra> annotation with key in the
	intent's \<@IntentMap> annotation. 


\end{itemize}
For example:

\<@IntentMap(@Extra(key="key",source=({A,B}),sink=({C,D})))
	Intent intent>
	
\begin{itemize}
\item
intent.putExtra("key",\<@Source({A}) @Sink({C,E})> obj value) is allowed
\item
intent.putExtra("key",\<@Source({A,B,E}) @Sink({C})> obj value) is not allowed

\end{itemize}

\subsubsection{getExtra}
The return type of intent.getExtra(key,defaultValue) is the type of the \<@Extra>
annotation with key in the intent's \<@IntentMap> annotation.
Calls intent.getExtra(key,defaultValue) type-check if:
\begin{itemize}
\item
key is in an \<@Extra> in the intent's \<@IntentMap> annotation.
\end{itemize}
For example:
		
\<@IntentMap(@Extra(key="key",source=({A,B}),sink=({C,D})))
		Intent getIntent()>
\begin{itemize}
\item
	Obj x = intent.getExtra("key", defaultValue) is allowed. 
	The type of x is \<@Source(A,B) @Sink(C,D) Obj>.
\item
	Obj x = intent.getExtra("key2", defaultValue) is not allowed.
\end{itemize}

\subsection{Type-checking sendIntent()}
An \<@IntentMap(A)> is copyable to \<@IntentMap(B)> if every key in the set of
@Extra in \<@IntentMap(A)> can be found in the set of \<@Extra> in
\<@IntentMap(B)> and the type of value of the \<@Extra> entry with key \<K> in
\<@IntentMap(B)> is a subtype of the type of value with key \<K> in
\<@IntentMap(A)>.


A call sendIntent() only type-checks if the sender's Intent type is copyable to
each possible receiver's Intent type according the component map. 

\section{Running the Intent Analysis with \TheFlowChecker{}}

Run \theFlowChecker{} in the command line with \<check-intent> instead of
\<check-flow> like the following command: 


\begin{alltt}
ant -DflowPolicy=myflowpolicy -DcomponentMap=mycomponentmap check-intent
\end{alltt}

If the Information Flow Checker produces any errors or warnings, then the app
has not been properly annotated and should be rejected.

%%  LocalWords:  sendIntent onReceive typeof callee
