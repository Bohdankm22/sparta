\htmlhr

\newcommand{\tp}[1]{\mathit{typeof}}
\newcommand{\Fix}[1]{\textbf{[[}{\color{red} #1}\textbf{]]}}
\newcommand{\theIntentChecker}{the Intent Checker\xspace}
\newcommand{\TheIntentChecker}{The Intent Checker\xspace}
\newcommand{\sendIntent}{\<sendIntent>}
\newcommand{\onReceive}{\<onReceive>}


\chapter{Intent Checker\label{intent-checker}}

This chapter describes \theIntentChecker{},
which type-checks information flows across communicating components of an
Android app.

Android intents are the standard inter-component communication
mechanism in Android. An Android component can send intents. They are used for
communication within an app, among apps, and with the Android system.
Intents can be seen as messages exchanged between components. Sensitive
data can flow in and out of intent objects. Consequently, to detect forbidden
flows derived from inter-component communication, \theIntentChecker{} aims to
correctly identify the information flow types of the data in an intent.

To use \theIntentChecker{}, a programmer must supply two types of information:
\begin{itemize}
\item
Information about inter-component communication across apps for the target
program. Section \ref{component-map} shows how to obtain and supply that
information.
\item
Type qualifiers used to express the data types in an intent. Section
\ref{intent-types} describes the type qualifiers used by \theIntentChecker.
\end{itemize}

Sending an intent is similar to an ordinary method call,
where the data in the intent are the method's arguments. 
\TheIntentChecker{} uses type qualifiers and inter-component communication
information to verify that for any sending intent method
call and its matching receiving intent method declarations,
the intent argument of the caller is compatible with that of the corresponding
callee. 

Intents can start an Activity calling the method \<startActivity>, a
Service calling the method \<startService>, or a BroadcastReceiver
calling \<sendBroadcast>. Intents are received in an Activity by the method
\<setIntent>, in a Service by the method \<onBind> or
\<onStartCommand> and in a BroadcastReceiver by the method
\<onReceive>. For simplicity in this manual, we abstract all methods
that send an intent as the method call \sendIntent{}, and all methods that
receive an intent as the method declaration \onReceive{}.

\section{Inter-component communication\label{component-map}}

To represent the inter-component communication of an app, we use a data
structure named component-map. Every app has a component map. The component map
of an app contains information regarding how the components of this app
communicate with each other, and how they communicate with components from other
apps. A component map file expresses a component map. The procedure to generate
a component map file for an app is detailed in section \ref{component-map-generation}.


\subsection{Semantics of a component map}
The component map helps
determining the call graph of an app by matching \sendIntent{} calls to
declarations of the \onReceive{} methods they implicitly invoke. A \sendIntent{}
call may be paired with more than one \onReceive{} declaration. One such pair
indicates that both components, possibly from different apps, may communicate
through intents. The set of pairs of communicating components is
conservative, that is it includes all possible pairs of methods that might
communicate.

\subsection{Syntax of a component map file}
Each line of a component map file specifies a set of receiver components for 
every intent sent from a particular method in the app.

For example, the following line

\code{com.package.ActivityA.foo() -> com.package.ActivityB,
com.package.ActivityC} 

\noindent
implies that every intent sent in the method \code{foo()} from 
\code{ActivityA} may be received by the components whose fully qualified class 
names are \code{com.package.ActivityB} and \code{com.package.ActivityC}.

\subsection{Using a component map file}
It is recommended to name the component map file as \code{component-map} and to
put it in the top level app directory, by doing so the ant target will use it
automatically when running:

\code{ant check-intent}

\noindent
Alternatively it is possible to pass the component map file path as an option
to ant:

\code{ant -DcomponentMap=mycomponentmapfilepath check-intent}

\section{Intent types\label{intent-types}}
An intent contains a map from string keys to arbitrary values. Consider an
intent variable \<i>. Data can be added to the map of \<i> by
the sender component with \<i.putExtra("key",val)> and then retrieved by the
receiver component with \<i.getExtra("key",default)>. An intent type is an
approximation to the keys that may be in the intent object at run time and to
the type of the values that those keys may map to. The type qualifiers used to
represent an intent type are \<@IntentMap> and \<@Extra>.

\subsection{Syntax}
The type qualifier \<@IntentMap> on an intent variable's type indicates the
types of the key/value mappings that are permitted to be accessed via
\<putExtra> and \<getExtra> calls.
An \<@IntentMap> type qualifier contains a set of \<@Extra> type qualifiers. An
\<@Extra> type qualifier contains a key \<K>, a source type \<SOURCE>, and a 
sink type \<SINK>. This means that the key \<K> maps to a value of source type 
\<SOURCE> and sink type \<SINK>. Consider the declaration below:

\begin{Verbatim}
@IntentMap({@Extra(key = "k1", source = {FILESYSTEM}, sink = {INTERNET}),
            @Extra(key = "k2", source = {FILESYSTEM}, sink = {DISPLAY})}) 
Intent i = new Intent();
\end{Verbatim}

The variable \<i> is annotated with an \<@IntentMap> type containing a set of
two \<@Extra> types, allowing this variable to be accessed via
\<i.putExtra("k1",val)>, \<i.putExtra("k2",val)>,
\<i.getExtra("k1",default)> and \<i.getExtra("k2",default)>. No other keys are
allowed to be accessed via \<putExtra> or \<getExtra> calls. 

Each intent variable's type must have only one \<@IntentMap> type qualifier, and
this type qualifier may have a finite set of \<@Extra> type qualifiers. Two
different \<@Extra> type qualifiers in the same \<@IntentMap> may not have the
same key \<K>.

For local variables, the type qualifiers \<@IntentMap> and \<@Extra> must be used
as cast types. Below is an example of an annotated local variable:

\begin{Verbatim}
void foo() {
    Intent i = (@IntentMap({
            @Extra(key = "k1", source = {FILESYSTEM}, sink = {INTERNET}),
            @Extra(key = "k2", source = {FILESYSTEM}, sink = {DISPLAY})}) Intent)
                new Intent();
}
\end{Verbatim}

Every \onReceive{} method has an intent formal parameter.
Below there is an example of an annotated intent formal
parameter for the \onReceive{} method \<setIntent>:

\begin{Verbatim}
@Override
public void setIntent( @IntentMap({
        @Extra(key = "location", source = {ACCESS_FINE_LOCATION }, sink = {})})
        Intent newIntent) {
    super.setIntent(newIntent);
}
\end{Verbatim}

\subsection{Semantics}
If variable \<i> is declared to have an intent type \textit{T}, then two facts must be
true. (1) All keys accessed at run-time by the value of variable \<i> must be
listed in \textit{T}. That is, the keys of the value of \<i> which are accessed
are a subset of \textit{T}'s keys. It is
permitted for the run-time value of variable \<i> to have fewer keys than those
listed by its type. It is also permitted for the run-time value of variable
\<i> to have more keys than those listed by its type but they may not be
accessed. (2) For any key \<k> that is accessed at run-time by the value of
variable \<i>, the type of its value is the type mapped by \<k> in \textit{T}. This can be more
concisely expressed as $\forall k.i[k] <: T[k]$. As permitted by object-oriented
typing, the run-time class of \<i[k]> may be a subtype of \textit{T}.

As an example, consider the declarations and method calls below:

\begin{Verbatim}
@IntentMap({@Extra(key = "picture", source = {FILESYSTEM}, sink = {INTERNET}),
            @Extra(key = "location", source = {FILESYSTEM}, sink = {DISPLAY})}) 
Intent i = new Intent();

@Source(FILESYSTEM) @Sink(INTERNET) File getPicture() {...}
@Source(ACCESS_FINE_LOCATION) @Sink(INTERNET) String getLocation() {...}

void fillIntent() {
    i.putExtra("picture", getPicture());        // Legal.
    i.putExtra("someRandomKey", getPicture());  // Violates requirement (1).
    i.putExtra("location", getLocation());      // Violates requirement (2).
    ...
}

void processDataFromIntent() {
    // pic will have source type FILESYSTEM and sink type INTERNET.
    File pic = i.getExtra("picture", null);               // Legal.
    // loc will have source type FILESYSTEM and sink type DISPLAY.
    String loc = i.getStringExtra("location", null);      // Legal.
    Object randomObject = i.getExtra("someRandomKey");    // Violates requirement (1)
    ...
}
\end{Verbatim}

\noindent
The type of variable \<i> indicates that this object must contain up to two
elements in its map which
will be accessed, one with key \<"picture">, source type \<FILESYSTEM>, and sink type
\<INTERNET>, and another with key \<"location">, source type \<FILESYSTEM>, and sink
type \<DISPLAY>. This object may contain more elements but they cannot be
accessed. The method calls in the method \<fillIntent> shows that it is only
valid to invoke putExtra if the value passed as argument is a subtype of the
declared type for the corresponding key. In the method \<processDataFromIntent>,
the variables \<pic> and \<loc> will have their source and sink types
inferred from the type of \<i>. Trying to access key \<randomObject> violates
requirement (1).


\subsubsection{Subtyping}
Intent type \<T1> is a subtype of intent type \<T2> if the key set of 
\<T2> is a subset of the key set of \<T1> and, for each key \<k> in both 
\<T1> and \<T2>, \<k> is mapped to the exact same type, that is, 
\<T1[k]> = \<T2[k]>.

\subsubsection{sendIntent calls and copyable rule}
A \sendIntent{} call is a call to one or more \onReceive{} methods.
A \sendIntent{} call type-checks if its intent argument is
copyable to the formal parameter of each corresponding \onReceive{} methods.
Copyable is a subtyping-like relationship with the weaker requirement: 
\<T1[k]> <: \<T2[k]> instead of \<T1[k]> = \<T2[k]>.
This is sound because the Android system passes a copy of the intent argument to
\onReceive{}, so aliasing is not a concern.

\subsubsection{Overriding and calling onReceive methods\label{override-onreceive}}
Every \onReceive{} method has a parameter of type \<Intent>, and this
parameter must be annotated with \<@IntentMap> and \<@Extra>.

The normal Java overriding rules do not apply to declarations of \onReceive{}. The
type of the formal parameter of \onReceive{} is not restricted by the type of the
parameter in the overridden declaration. This is allowable because by convention
\onReceive{} is never called directly but rather is only called by the Android
system via a \sendIntent{} method call. \TheIntentChecker{} prohibits direct
calls to \onReceive{} methods.

There is a particularity for the \onReceive{} method in Activities, \<setIntent>.
Every Activity that calls the method \<getIntent> must override both methods
\<setIntent> and \<getIntent>. The return type of \<getIntent> must be annotated
with the same type as the formal parameter of \<setIntent>, so that when the
method \<getIntent> is called the correct type is returned.

%%  LocalWords:  \sendIntent{} \onReceive{} typeof callee
