


\section{Flow Policy\label{sec:flow-policy}}

A flow policy is a list of all the information flows that are permitted to occur
in an application.
A flow policy file expresses a flow policy, as a list of $\flow{\|flowsource|^*}{\|flowsink|^*}$ pairs.
Just as the Android manifest lists all the permissions that an app uses,
the flow policy file lists the flows among permissions and other sensitive 
locations. 

\subsection{Semantics of a Flow Policy}
\label{sec:undsiredflows}
The Information Flow Checker guarantees that there is no information
flow except for what is explicitly permitted by the policy file. If a user writes a type that is
not permitted by the policy file, then the Information Flow Checker issues a warning
even if all types in program otherwise typecheck.

For example, this variable declaration

\begin{alltt}
@Source(CAMERA) @Sink(INTERNET) Video video = ...
\end{alltt}

\noindent
is illegal unless the the policy file contains:

\begin{alltt}
CAMERA -> INTERNET
\end{alltt}

Here is another example.
The flow policy file contains:
\begin{alltt}
  ACCOUNTS      -> EXTERNAL_STORAGE, FILESYSTEM
  ACCELEROMETER -> EXTERNAL_STORAGE, FILESYSTEM, INTERNET
\end{alltt}

The following variable declarations are permitted:
\begin{alltt}
  @Source(ACCOUNTS) @Sink(EXTERNAL_STORAGE) Account acc = ...
  @Source({ACCELEROMETER, ACCOUNTS})
  @Sink({EXTERNAL_STORAGE, FILE_SYSTEM}) int accel = ...
\end{alltt}

The following definitions would generate ``forbidden flow'' errors:
\begin{Verbatim}
  @Source(ACCOUNTS) @Sink(@INTERNET) Account acc = ...
  @Source({ACCELEROMETER, ACCOUNTS})
  @Sink({EXTERNAL_STORAGE, FILESYSTEM, INTERNET})
\end{Verbatim}

\myparagraph{Transitivity and the flow policy file}
\label{sec:flow-policy-transitivity}
The flow policy file indicates specific permitted information flows.  It
may be possible to combine these flows.
For example, a policy that permits
\flow{CAMERA}{FILESYSTEM} and \flow{FILESYSTEM}{INTERNET}
will implicitly allow the flow \flow{CAMERA}{INTERNET},
because the application may record from the camera into a file
and then send the contents of the file over the network.
\TheFlowChecker forbids such implied flows:  the developer is required to write
the transitive flow in the flow policy file, which requires the developer
to justify its purpose or convince the app store that the flow is not used.
%Finer-grained permissions, as discussed in
%\secref{sec:future}, would reduce the impact of implied flows.
\todo{Explain/show warning}


\subsection{Syntax of a Flow Policy File}

Each line of a policy file specifies a permitted flow from a source to one
or more sinks.  For example,
\code{MICROPHONE -> INTERNET} implies that
MICROPHONE data is always allowed to flow to INTERNET.
The source or sink must be a member of the enum
\<sparta.checkers.quals.FlowPermission>.  
ANY is allowed just as it is in @Source and @Sink, but empty, \ttcbs, is not allowed.

Multiple sinks can appear on the same line if they are separated by commas. 
For example,
this policy file:
\begin{alltt}
   MICROPHONE -> INTERNET, LOG, DISPLAY
\end{alltt}
is equivalent to this policy file:
\begin{alltt}
   MICROPHONE -> INTERNET
   MICROPHONE -> LOG
   MICROPHONE -> DISPLAY, INTERNET
\end{alltt}

The policy file may contain blank lines and comments that begin with 
a number sign (\<\#>) character.



\subsection{Using a flow-policy file}
To use a flow-policy file when invoking the Information Flow Checker from the
command line, pass it the option:
\begin{alltt}
-AflowPolicy=\emph{mypolicyfile}
\end{alltt}

Or if you are using the \<check-flow> ant target, you can pass the option to ant:
\begin{alltt}
ant -DflowPolicy=\emph{mypolicyfile} check-flow
\end{alltt}
If the flow policy is named \<flow-policy> and is located the top level app directory, the ant 
target will use it automatically.


