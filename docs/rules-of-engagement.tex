\htmlhr
\chapter{Requirements of the app developer (rules of engagement)\label{rules-of-engagement}}

The goal of an application developer is to create a safe, functional
application --- and to write the documentation and code so that the safety
and functionality are immediately obvious.  In particular, the code and
documentation should be clear and complete, and the system should pass all
the tests that the SPARTA toolset performs.  If the application developer
fails to meet any of these objectives, then the application will be
rejected from the app store, and the fault will be with the application
developer, not with the app store.

A malicious developer would need to write clear documentation and code, but
would attempt to hide malicious behavior in the app nonetheless.  If the
documentation or code is not clear, or if the malicious behavior is not
well-hidden, or if the SPARTA tools do not confirm that the code conforms
to the documentation, then the malicious developer has failed in his task.

Note that the malicious developer's goal is more difficult than just
writing malicious code, and is even more difficult than writing well-hidden
malicious code.  The reason is that the SPARTA toolset encourages good
coding style:  poor style requires more warning suppressions.
The SPARTA tools lead a programmer to better, clearer code.


Here are some specific requirements of the app developer:

\begin{itemize}
\item
  Use good style.  Code must not be obfuscated.  Raw types must not be
  used.  Minimize use of undesirable/unsound features such as arrays,
  casts, heterogeneous collections, and reflection.

  Provide source code.  Provide a build file (for Ant, Maven, Android,
  etc.).

\item
  State the intended information flows in the application.  This should
  be expressed both in precise English and also in a machine-readable
  format that can be read by the SPARTA tools.

  The English description should include how the information flows between
  parts of the application (the paths along which information flows), and
  the conditions under which it flows (such as only after a particular user
  action or external trigger).  These will eventually be represented in the
  SPARTA toolset's file format and checked by SPARTA, but they are not yet.

\item
  Annotate the application.  Write type qualifiers on variables.  This is
  essentially just a restatement of the information flows above at a lower
  and more detailed level.

\item
  Type-check the application.  Run the type-checker on its source code.  Do
  not take advantage of any of the unsound features of the type-checker.
  (Those features are supported to reduce the workload of people who are
  not concerned about an absolute guarantee.)  For example, do not skip any
  portions of the code

\item
  Minimize the number of type-checking failures, and justify each one.  A
  type-checking failure indicates either a bug (i.e., security flaw) in the
  application, or an instance of subtle code that is beyond the
  capabilities of the type system.  In either case, the app developer's
  first inclination should be to rewrite the code to be correct and
  straightforward.

  If rewriting the code is impossible, then every remaining warning should
  be suppressed with a \<@SuppressWarnings> annotation.  Every
  \<@SuppressWarnings> annotation requires a clear, compelling
  justification regarding why the code is actually correct and safe (even
  though the type-checker cannot prove this property), and why the code
  cannot be rewritten to address the warning.  This justification should be
  written in the source code at the location of the \<@SuppressWarnings>
  annotation.

  An excessive number of type-checker warnings, or missing justifications
  for warning suppressions, is grounds for rejection from the app store.

\end{itemize}



%%% Local Variables: 
%%% mode: latex
%%% TeX-master: "manual"
%%% TeX-command-default: "PDF"
%%% End: 

%  LocalWords:  app toolset's
