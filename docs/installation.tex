\htmlhr
\chapter{Installation and app setup\label{installation}}
This chapter describes how to install the SPARTA tools
(Section~\ref{sec:install}) and how to prepare an Android app to run the
SPARTA tools. (Section~\ref{sec:antsetup}).

\paragraph{Checker Framework}
\begin{itemize}
\item If you are using the released version of SPARTA, follow the installation instructions in the manual: 
\url{http://types.cs.washington.edu/checker-framework/current/checkers-manual.html#installation}
\item If you are using the development version of SPARTA,  follow Section 25.3 from the Checker Framework manual (Building from source.)
\item  For both versions, as described in the installation instructions, set the \<CHECKERFRAMEWORK>
  environment variable to \<.../checker-framework/>
\end{itemize}

\section {Requirements\label{sec:requirements}}
\paragraph{Java 7}
\begin{itemize}
 \item  \<.../jdk1.7.0/bin> must be on your path.
 \item \<JAVA\_HOME> should be set to \<.../jdk1.7.0>.
\end{itemize}

\paragraph{Ant}
\begin{itemize}
 \item Ant version 1.8.2 or later
\end{itemize}

\paragraph{Android SDK}
\begin{itemize}
 \item Android API version 15 or later
\end{itemize}
\begin{enumerate}
 \item Install the Android SDK to some directory. 
 \item Set \<ANDROID\_HOME> to the directory where you installed the
   Android SDK.
 \item Download the \<android-15> target by running \<\$ANDROID\_HOME/tools/android>
\end{enumerate}

If using Eclipse, go to
\<Help $\rightarrow$ Install New Software>
and install the Android ADT Plugin (\url{https://dl-ssl.google.com/android/eclipse}).




\section{Install SPARTA\label{sec:install}}

\begin{enumerate}

\item
  Obtain the source code for the SPARTA tools, either from its version
  control repository or from a packaged release.

\begin{itemize}
\item
  To obtain from the version control repository, run
\begin{Verbatim}
 hg clone https://dada.cs.washington.edu/hgweb/sparta-code
\end{Verbatim}
  using the credentials you have been given.
\item 
  If you do not have access to the source code repository, then
  download the SPARTA release from
  \url{http://types.cs.washington.edu/sparta/release/}.  

  Then, unpack the archive.
\end{itemize}

%
%Rationale:  When working with Android, a developer must ``update a project'' to
%properly set the path to the Android SDK\@.  For more details about
%updating an Android project, see
%\url{http://developer.android.com/tools/projects/projects-cmdline.html#UpdatingAProject}.


\item Build the SPARTA tools by compiling the source code:
\begin{alltt}
ant jar
\end{alltt}

\item
As a sanity check of the installation, run

\begin{alltt}
ant all-tests
\end{alltt}

You should see ``\<BUILD SUCCESSFUL>'' at the end.


\end{enumerate}


\section{Android App Setup\label{sec:antsetup}}

This section explains how to set up an Android application for analysis with the SPARTA tools.
\begin{enumerate}

\item
Ensure the following environment variables are set. 

\begin{itemize}
\item
\<CHECKERFRAMEWORK> is the
\<.../checker-framework/> directory

\item
\<SPARTA\_CODE> is the \<.../sparta-code> directory

\item
\<ANDROID\_HOME> is the \<.../android-sdk> directory

\end{itemize}

\item
Update the app configuration by running the following command in the main directory of the
app.
\begin{Verbatim}
$ANDROID_HOME/tools/android update project --path .
\end{Verbatim}

\item Build the app

\begin{Verbatim}
ant release
\end{Verbatim}
If the app does not build with the above commend, then the SPARTA tool set will 
not be able to analyze the app.  Below are two common compilation issues are solutions.

Most Android apps will rely on an auto-generated \<R.java> file
in the \</gen> directory of the project. This will only be generated
if there are no errors in the project. There may be errors in the
resources (\<.../res> directory) that could cause \<R.java> to not be
generated.

Additionally, if the app depends on an external \<.jar> file (often
located in the \<lib/> directory), it will compile in Eclipse but not
with Ant. To fix this, in ant.properties, add ``\<jar.libs.dir=lib>''
(or wherever the \<.jar> is located).





\item
Add the SPARTA build targets to the end of the \<build.xml>
 file, just above \verb|</project>|.


\begin{Verbatim}
    <property name="checker-framework" value="${env.CHECKERFRAMEWORK}"/>
    <property name="sparta-code" value="${env.SPARTA_CODE}"/>

    <dirname property="checker-framework_dir" file="${checker-framework}/checker-framework" />
    <dirname property="sparta-code_dir" file="${sparta-code}/sparta-code" />

    <import file="${sparta-code_dir}/build.include.xml" />
    <property name="flowPolicy" value="flow-policy"/>
    <property name="componentMap" value="component-map"/>
\end{Verbatim}

\end{enumerate}

\subsubsection{Using Eclipse to analyze apps}
To use Eclipse to look at and build the code, perform these 
steps:
\begin{itemize}
\item
Import the projects the app.      \<Import $\rightarrow$ Existing Android Code 
Into Workspace>

    \item
    Make sure
    \<Project Properties $\rightarrow$ Android $\rightarrow$ Android
    version \#> is checked

    \item
    Check that
    \<Project Properties $\rightarrow$ Java Build Path $\rightarrow$
    Libraries $\rightarrow$ Android version \#> appears

    \item
    Add sparta.jar to the apps build path
    
    \item Right click on the build.xml file and select   \<Run as $\rightarrow$ 
    External Tools Configurations...>. In the \<Main> tab add \<check-flow> to the 
    Arguments box.  In the \<Environment> tab, add the \<CHECKERFRAMEWORK> and 
    \<SPARTA\_CODE> variables.  
    
  \end{itemize}





%%% Local Variables: 
%%% mode: latex
%%% TeX-master: "manual"
%%% TeX-command-default: "PDF"
%%% End: 

%  LocalWords:  hg SDK ADT Plugin MercurialEclipse gson json sparta jsr308
%  LocalWords:  Dandroid langtools javaparser app TODO xml env dirname dir
%  LocalWords:  basename
