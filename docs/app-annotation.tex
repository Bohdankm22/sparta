\htmlhr
\chapter{How to Annotate an Android App\label{app-annotation}}

This chapter describes the best practices for annotating an Android application.  In general, only 
fields, methods, and APIs need to been annotated.

\section{Flow Checker}
An application is fully annotated when the \<ant flowtest> returns no errors.  So the typical work flow is 
to run \<flowtest>, add a few annotations or annotate a few API methods, and the run \<flowtest> again.  
This repeats until all of the annotation errors have been resolved.

\section{Typical Errors}

\paragraph{Forbidden Flow}  To correct a forbidden flow error, add the flow or flows to the policy file.  
For example, fix the error below by adding \<LITERAL -\textgreater FILESYSTEM> to the policy file.
\begin{Verbatim}
NewTest.java:43: error: flow forbidden by flow-policy  
        test = new @FlowSinks(FlowSink.FILESYSTEM)@FlowSources(FlowSource.LITERAL) TestClass2(fs);
                  ^
  found: @FlowSinks(FlowSink.FILESYSTEM) @FlowSources(FlowSource.LITERAL) TestClass2 
  forbidden flows:
    LITERAL -> FILESYSTEM
\end{Verbatim}

\paragraph{Conservative Typing}

APIs that have not been annotated have been typed so conservatively that they will always produce incompatible types errors where the required is \<@FlowSinks(ANY) @FlowSources({})> or 
\<@FlowSinks({}) @FlowSources(ANY)>.  These errors can be fixed by annotating the API method; 
Section \ref(sec:annotatedAPI) gives guidelines for annotating APIs. 
Below is an example of this sort of error.

\begin{Verbatim}
HelloWorld.java:84: error: incompatible types in argument.
                 .replace(R.id.container, fragment)
                                          ^
   found   : @FlowSinks(CONDITIONAL) @FlowSources(LITERAL) Fragment
   required: @FlowSinks(ANY) @FlowSources({}) Fragment
\end{Verbatim}

\paragraph{Incompatible Types}

If the incompatible types error is not from conservative defaulting, then the error must be fixed by adding or
removing annotations in the application.  For example, the error below can be fixed by adding ACCELEROMETER to the FlowSource of the return type.  

\begin{Verbatim}
HelloWorld.java:49: error: incompatible types in return.
       return x;
              ^
   found   : @FlowSinks(CONDITIONAL) @FlowSources({LITERAL, ACCELEROMETER}) int
   required: @FlowSinks(CONDITIONAL) @FlowSources(LITERAL) int
\end{Verbatim}

\paragraph{Conditionals}
Any item in a conditional statement must have CONDITIONAL listed as a FlowSink.  If a variable is only annotated with FlowSources and strict conditionals are not used, then CONDITIONAL is added as a flow sink by default.  For example, if input is a parameter in a method and is annotated with @FlowSinks(FlowSink.NETWORK), the following error will occur.  To fix the error, add CONDITIONAL to the flow sink annotation.  

\begin{Verbatim}
HelloWorld.java:48: error: Conditions are not allowed to depend on flow information.
         if( i1 > 2){
            ^
\end{Verbatim}

\paragraph{StubParser}

Errors like the one below are from the parsing of the stub file.  They happen because of typos in flow.astub or  
if an API does not exist in the version of Android or Java used.

\begin{Verbatim}
StubParser: Method isLLowerCase(char) not found in type java.lang.Character
\end{Verbatim}
This error can be corrected by removing the extra L in the method name in flow.astub.

\section{Annotating Application Methods}

Typically, return types should be annotated with just FlowSources so that the FlowSinks can be inferred from the policy file as explained in Section \ref{flow-defaults}.
Similarly, parameters should only be annotated with FlowSinks, so that the FlowSources can be inferred from  the policy file.  Local variables should not have to be annotated, their types can be inferred. Fields must be annotated with at FlowSinks or FlowSources, or sometimes both. 
The guidelines for annotating API methods described in the next sections also apply to annotating methods in your application. 


\section{Annotating APIs\label{sec:annotatedAPI}}

File \<sparta-code/src/sparta/checkers/flow.astub> provides Android and Java
library annotations.  You may need to enhance it, if you find that your application
uses APIs that are not yet annotated.  (Remember to rebuild sparta.jar each time you change flow.astub)

Also, see the manual 
\url{http://types.cs.washington.edu/checker-framework/current/checkers-manual.html#annotating-libraries}

For every API method used by this app (i.e., those output by \<ant
reportapiusage>) that does \emph{not} already appear in file
\<sparta-code/src/sparta/checkers/flow.astub>, read the Javadoc and decide what flow properties the method
has.  Use this information to annotate the method in \<flow.astub>.  The next few sections have guidelines on
how to annotate certain kinds of methods. 

To speed up annotating an entire API class, copy the output from Checker Framework's StubGenerator to 
flow.astub and then annotate the methods.  See \url{http://types.cs.washington.edu/checker-framework/current/checkers-manual.html#stub} for more details. 

\subsection{Methods with Sources}
If you read the javadoc and the method returns an object from a certain source, then you should annotated 
the return with that flow source and an ANY flowsink.

The getParameters method is an example of a method that returns an object with a source.
\begin{Verbatim}
package android.hardware;
class Camera{
    @FlowSources(FlowSource.CAMERA_SETTINGS)@FlowSinks(FlowSink.ANY)
        Camera.Parameters getParameters ();
    void setParameters (@FlowSources(FlowSource.ANY)@FlowSinks(CAMERA_SETTINGS) 
         Camera.Parameters params);
}
\end{Verbatim}

\begin{Verbatim}
Camera.Parameters parameters
     = mCamera.getParameters();
//Change some parameters
//If the policy file contains: CAMERA_SETTINGS->CAMERA_SETTINGS
mCamera.setParameters(parameters);
\end{Verbatim}



\subsection{Methods with Sinks}
If you read the javadoc for an API method an it sends some parameters to a FlowSink, then the parameters
should be annotated with that flow sink and ANY flow source.  

Example annotation in flow.astub
\begin{Verbatim}
package android.database.sqlite;
class SQLiteDatabase{
  public void execSQL (@FlowSinks(FlowSink.DATABASE) @FlowSource(FlowSource.ANY) String sql);
}
\end{Verbatim}
Use of the API
\begin{Verbatim}
@FlowSources(SMS,LITERAL)  String getSMSQuery(){..}
String mes = getSMSQuery();
//if SMS,LITERAL->DATABASE is in flow policy
db.execSQL(mes);
\end{Verbatim}

\subsection{Callbacks}
The Android API frequently uses callbacks.  These are methods that the developer must 
implement and register. In flow.astub, these callbacks should be annotated
with source information and FlowSink.ANY.  

An example annotation of a callback method
\begin{Verbatim}
package android.hardware;
class Camera$PictureCallback{
  //data: a byte array of the picture data
   void onPictureTaken 
   (@FlowSource(CAMERA)@FlowSinks(FlowSink.ANY) byte[] data, 
   @FlowSource(CAMERA)@FlowSinks(FlowSink.ANY)  Camera camera);
}
\end{Verbatim}

An example implementation of a callback
\begin{Verbatim}
public void onPictureTaken
(@FlowSource(CAMERA) byte[] data, @FlowSource(CAMERA) Camera camera){
   //If CAMERA->FILE_SYSTEM is in policy file
    writeToFile(data);
}
\end{Verbatim}


\subsection{Methods that Transform Data}

Some methods take the arguments passed, transform them, and then return them.  These sorts of 
methods should be annotated with \<@PolyFlowSources @PloyFlowSinks>
  to preserve the flow information.  The declaration annotation \<@PolyFlow> can be used instead of
  annotating all the parameters and return types. See Section \ref{sec:polyflow} for more information 
  
  Math.min(...) is a good example of these kinds of methods. 
  
  \begin{Verbatim}
package java.lang;
class Math{
   @PolyFlow  
   int min(int i1, int i2);
}
\end{Verbatim}

Example use of \<@PolyFlow>.
\begin{Verbatim}
@FlowSources(FlowSource.LOCATION) int i1 = getLocation();
@FlowSources(FlowSource.NETWORK) int i2 = getLocationForNetwork();
@FlowSources({FlowSource.LOCATION,FlowSource.NETWORK)}) int min = Math.min(i1,i2);
 \end{Verbatim}


