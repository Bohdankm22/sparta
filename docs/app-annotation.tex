\htmlhr
\chapter{Tips for writing information flow annotations\label{app-annotation}}

This chapter contains tips for annotating an Android application.  The Checker
Framework has some general tips, too.  See \url{http://types.cs.washington.edu/checker-framework/current/checker-framework-manual.html#tips-about-writing-annotations} for those tips.

In general, only fields and methods signatures in your own code and in
libraries need to be annotated. Usually method bodies do not need to be
annotated.


\section{Annotating  methods\label{sec:annomethods}}

Typically, return types should be annotated with just \<@Source> so that \<@Sink> can be
 inferred from the policy file as explained in \secref{sec:infer-from-flow-policy}. Similarly, parameters should
  only be annotated with  \<@Sink>, so that the \<@Source> can be inferred from  the policy file.
    Local variables should not have to be annotated, because their types can be inferred. Fields 
    must be annotated with   \<@Sink> or \<@Source>, or sometimes both. 

\section{Annotating API methods in stub files\label{sec:annoAPI}}

\subsection{Callbacks}
The Android API frequently uses callbacks, that the developer  
implements in an implementing class  and then registers through some API.
 In stub files, these callbacks should be annotated
with source information that will be passed when the method is called.  

An example annotation of a callback method
\begin{Verbatim}
package android.hardware;
class Camera$PictureCallback{
  //data: a byte array of the picture data
   void onPictureTaken (@Source(CAMERA) byte[] data, Camera camera);
}
\end{Verbatim}

An example implementation of a callback
\begin{Verbatim}
public void onPictureTaken(@Source(CAMERA) byte[] data, Camera camera){
   //If CAMERA->FILE_SYSTEM is in policy file
   //then the following statement will not give an error
    writeToFile(data);
}
\end{Verbatim}


\subsection{Methods that transform data}

Some methods take the arguments passed, transform them, and then return them.  These sorts of 
methods should be annotated with \<@PolySource @PloySink>
  to preserve the flow information.  The declaration annotation \<@PolyFlow> can be used instead of
  annotating all the parameters and return types. See \secref{sec:polyflowsources} for more information. 
  
  \<Math.min(...)> is a good example of these kinds of methods. 
  
  \begin{Verbatim}
package java.lang;
class Math{
   @PolyFlow  
   int min(int i1, int i2);
}
\end{Verbatim}

Example use of \<@PolyFlow>.
\begin{Verbatim}
@Source(LOCATION) int i1 = getLocation();
@Source(INTERNET) int i2 = getLocationForNetwork();
@Source({LOCATION,INTERNET)}) int min = Math.min(i1,i2);
 \end{Verbatim}

\section{Common errors\label{errors}}

This section explains some common errors issued by the Information Flow Checker, and
gives advice about correcting the errors.   

Also see the Checker Framework Manual
(\ifhevea\url{http://types.cs.washington.edu/checker-framework/current/checker-framework-manual.html}\else\url{http://types.cs.washington.edu/checker-framework/current/checker-framework-manual.pdf}\fi),
which contains information about pluggable type-checking in general.  Many
of your errors may not be specific to the Information Flow Checker and are likely to be
answered in the Checker Framework Manual.

If you encounter a problem you cannot solve, contact the SPARTA developers (\secref{sec:incaseoftrouble}).

\paragraph{Incompatible Types}

In general, 
Information Flow Checker warnings are written as shown below.

\begin{Verbatim}  
../SomeClass.java:line number: warning: some types are incompatible 
             source code causing warning
                     a caret (^) pointing to the location of the warning.
   found   : Qualified type found by the Information Flow Checker
   required: Qualified type that the Information Flow Checker was expecting.
\end{Verbatim}
 
In order to correct a warning and correctly capture the app behavior, answer 
the following questions for each warning.
\begin{enumerate}
\item Why are the found and required annotations those listed in the warning message?
   \begin{itemize}
    \item Is the annotation explicitly written in the source code? 
    \item Is the annotation from an API method that was annotated in stub file? \secref{sec:apispecs}
    \item Is the annotation inferred? \secref{sec:type-inference}
    \item Is the annotation defaulted? \secref{sec:unannotated-types}
    \end{itemize}
\item Why is the found type not a subtype of the required type? \secref{sec:subtyping}
  \begin{itemize}
   \item Does the found type have more or different found sources than required?
   \item Does the found type have less or different found sinks than required? 
   \end{itemize}
\item What annotation or annotations correctly capture the behavior of the app at
  this location? (In other words, what annotation will make the found type a 
  subtype of the required type?)
   \begin{itemize}
   \item Add only a source or a sink annotation
   \end{itemize}
\end{enumerate}

\subsection{Forbidden flow}  
Every source-sink pair in your code must be listed in the flow policy or else a \emph{forbidden flow} error will occur.
To correct a forbidden flow error, add the forbidden flow to the policy file. 
  
For example, the warning below can be corrected by adding  \code{LITERAL -> FILESYSTEM} to the policy file, if this flow is justified and innocuous. 
\begin{Verbatim}
NewTest.java:43: error: flow forbidden by flow-policy  
        test = new @Sink(FlowPermission.FILESYSTEM)@Source(FlowPermission.LITERAL) TestClass2(fs);
                  ^
  found: @Sink(FlowPermission.FILESYSTEM) @Source(FlowPermission.LITERAL) TestClass2 
  forbidden flows:
    LITERAL -> FILESYSTEM
\end{Verbatim}




%%% Local Variables: 
%%% mode: latex
%%% TeX-master: "manual"
%%% TeX-command-default: "PDF"
%%% End: 

%  LocalWords:  App check-flow Sink Source sparta app FlowPermission
%  LocalWords:  reportapiusage StubGenerator getParameters PolySource
%  LocalWords:  PloySink PolyFlow FILESYSTEM
