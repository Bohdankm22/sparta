\htmlhr
\chapter{Tips for writing information flow annotations\label{app-annotation}}

This chapter contains tips for annotating an Android application.  The Checker
Framework has some general tips, too.  See \url{http://types.cs.washington.edu/checker-framework/current/checkers-manual.html#tips-about-writing-annotations} for those tips.

In general, only fields and methods signatures in your own code and in
libraries need to be annotated. Usually method bodies do not need to be
annotated.


\section{Annotating  Methods\label{sec:annomethods}}

Typically, return types should be annotated with just \<@Source> so that \<@Sink> can be
 inferred from the policy file as explained in \secref{sec:infer-from-flow-policy}. Similarly, parameters should
  only be annotated with  \<@Sink>, so that the \<@Source> can be inferred from  the policy file.
    Local variables should not have to be annotated, because their types can be inferred. Fields 
    must be annotated with   \<@Sink> or \<@Source>, or sometimes both. 

\section{Annotating API Methods in Stub Files\label{sec:annoAPI}}

\subsection{Callbacks}
The Android API frequently uses callbacks, that the developer  
implements in an implementing class  and then registers through some API.
 In stub files, these callbacks should be annotated
with source information that will be passed when the method is called.  

An example annotation of a callback method
\begin{Verbatim}
package android.hardware;
class Camera$PictureCallback{
  //data: a byte array of the picture data
   void onPictureTaken (@Source(CAMERA) byte[] data, Camera camera);
}
\end{Verbatim}

An example implementation of a callback
\begin{Verbatim}
public void onPictureTaken(@Source(CAMERA) byte[] data, Camera camera){
   //If CAMERA->FILE_SYSTEM is in policy file
   //then the following statement will not give an error
    writeToFile(data);
}
\end{Verbatim}


\subsection{Methods that Transform Data}

Some methods take the arguments passed, transform them, and then return them.  These sorts of 
methods should be annotated with \<@PolySource @PloySink>
  to preserve the flow information.  The declaration annotation \<@PolyFlow> can be used instead of
  annotating all the parameters and return types. See \secref{sec:polyflowsources} for more information. 
  
  \<Math.min(...)> is a good example of these kinds of methods. 
  
  \begin{Verbatim}
package java.lang;
class Math{
   @PolyFlow  
   int min(int i1, int i2);
}
\end{Verbatim}

Example use of \<@PolyFlow>.
\begin{Verbatim}
@Source(LOCATION) int i1 = getLocation();
@Source(INTERNET) int i2 = getLocationForNetwork();
@Source({LOCATION,INTERNET)}) int min = Math.min(i1,i2);
 \end{Verbatim}

\section{Common Errors\label{errors}}

This section explains some common errors issued by the Information Flow Checker, and
gives advice about correcting the errors.   

Also see the Checker Framework Manual
(\ifhevea\url{http://types.cs.washington.edu/checker-framework/current/checkers-manual.html}\else\url{http://types.cs.washington.edu/checker-framework/current/checkers-manual.pdf}\fi),
which contains information about pluggable type-checking in general.  Many
of your errors may not be specific to the Information Flow Checker and are likely to be
answered in the Checker Framework Manual.

If you encounter a problem you cannot solve, contact the SPARTA developers (\secref{sec:incaseoftrouble}).



\subsection{Forbidden Flow}  
Every source-sink pair in your code must be listed in the flow policy or else a \emph{forbidden flow} error will occur.
To correct a forbidden flow error, add the forbidden flow to the policy file. 
  
For example, the warning below can be corrected by adding  \code{LITERAL -> FILESYSTEM} to the policy file, if this flow is justified and innocuous. 
\begin{Verbatim}
NewTest.java:43: error: flow forbidden by flow-policy  
        test = new @Sink(FlowPermission.FILESYSTEM)@Source(FlowPermission.LITERAL) TestClass2(fs);
                  ^
  found: @Sink(FlowPermission.FILESYSTEM) @Source(FlowPermission.LITERAL) TestClass2 
  forbidden flows:
    LITERAL -> FILESYSTEM
\end{Verbatim}

\subsection{Incompatible Types}
The most common error type is \emph{incompatible types}.  They can be in arguments,  assignment, return, etc.

\paragraph{Conservative Typing}

APIs that have not been annotated have been typed so conservatively that they will always produce incompatible types errors where the required is \<@Sink(NOT\_REVIEWED) @Source(NOT\_REVIEWED)>.  These errors can be fixed by annotating the API method; 
Section \ref{flow-task-annotate-apis} explains how to annotate APIs. 
Below is an example of this sort of error.

\begin{Verbatim}
HelloWorld.java:84: error: incompatible types in argument
                 .replace(R.id.container, fragment)
                                          ^
   found   : @Sink(CONDITIONAL) @Source(LITERAL) Fragment
   required: @Sink(NOT_REVIEWED) @Source(NOT_REVIEWED) Fragment
\end{Verbatim}

\paragraph{Incompatible Types}

If the error is not because of an unannotated API, then the error must be fixed by adding or
removing annotations in the application.  For example, the error below can be fixed by adding ACCELEROMETER to the FlowPermission of the return type.  

\begin{Verbatim}
HelloWorld.java:49: error: incompatible types in return.
       return x;
              ^
   found   : @Sink(CONDITIONAL) @Source({LITERAL, ACCELEROMETER}) int
   required: @Sink(CONDITIONAL) @Source(LITERAL) int
\end{Verbatim}

\subsection{Conditionals}
As explained in \secref{sec:conditionals}, any item in a conditional statement must have CONDITIONAL listed as a FlowPermission.  If a variable is only annotated with \<@Source> and strict conditionals are not used, then CONDITIONAL is added as a flow sink by default. 

For example, if input is a parameter in a method and is annotated with @Sink(FlowPermission.INTERNET), the following error will occur.  To fix the error, add CONDITIONAL to the flow sink annotation.  

\begin{Verbatim}
.../HelloWorld.java:43: warning: sensitive source information found in a conditional.
     if(location.equals(WASHINGTON))
       ^
   found: [ACCESS_FINE_LOCATION]
\end{Verbatim}



%%% Local Variables: 
%%% mode: latex
%%% TeX-master: "manual"
%%% TeX-command-default: "PDF"
%%% End: 

%  LocalWords:  App check-flow Sink Source sparta app FlowPermission
%  LocalWords:  reportapiusage StubGenerator getParameters PolySource
%  LocalWords:  PloySink PolyFlow FILESYSTEM
